\chapter{Resumen}
\hyphenpenalty=10000
\tolerance=2000
\emergencystretch=10pt

\noindent
El presente trabajo desarrolla un análisis probabilista para la inferencia de eventos sísmicos significativos en la costa del Pacífico mexicano, región con una alta actividad sísmica. Utilizando datos históricos del Servicio Sismológico Nacional de México desde 1900 hasta julio de 2025, se implementó una metodología sistemática que comprende estadística descriptiva, intervalos de confianza, pruebas de hipótesis y pruebas de bondad de ajuste para seis regiones: Chiapas, Guerrero, Michoacán, Oaxaca, Resto Nacional y Sismos Nacionales. Los resultados revelan que las distribuciones Gumbel, Weibull, GEV, Logística y Normal ajustan adecuadamente los datos de Magnitudes Máximas anuales según la región analizada, mientras que ninguna distribución ajustó satisfactoriamente para los Sismos Totales. Se identificaron diferencias estadísticamente significativas entre las regiones en términos de media, varianza y proporción de sismos superiores a 6.5°. Este trabajo busca proporcionar una base robusta para la estimación probabilista de eventos sísmicos futuros en México y contribuir al desarrollo de políticas de prevención y concientización del peligro sísmico existente en México.


\vspace{0.9cm}
\textbf{Palabras clave:} Inferencia Sísmica, análisis probabilista, estadísticos descriptivos, pruebas de bondad de ajuste, pruebas de hipótesis, sismos en la costa del Pacífico Mexicano.

\cleardoublepage

\chapter{Abstract}
\hyphenpenalty=10000
\tolerance=2000
\emergencystretch=10pt  

\noindent
This study develops a probabilistic analysis for the inference of significant seismic events on the Mexican Pacific coast, a region with high seismic activity. Using historical data from the Mexican National Seismological Service from 1900 to July 2025, a systematic methodology was implemented. This methodology includes descriptive statistics, confidence intervals, hypothesis testing, and goodness-of-fit tests for six regions: Chiapas, Guerrero, Michoacán, Oaxaca, the rest of the country, and national seismic events. The results reveal that the Gumbel, Weibull, GEV, Logistic, and Normal distributions adequately fit the annual Maximum Magnitudes data, depending on the region analyzed. In contrast, no distribution satisfactorily fit the data for Total Earthquakes. Statistically significant differences were identified among the regions in terms of mean, variance, and the proportion of earthquakes with a magnitude greater than 6.5. This work aims to provide a robust basis for the probabilistic estimation of future seismic events in Mexico and to contribute to the development of prevention policies and awareness of the existing seismic hazard in the country.


\vspace{0.9cm}
\textbf{Keywords:} Seismic Inference, probabilistic analysis, descriptive statistics, goodness-of-fit test, hypothesis testing, earthquakes on the Mexican Pacific coast.