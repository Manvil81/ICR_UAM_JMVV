\chapter{Conclusiones \label{cap:Conclusiones}}

\noindent
El presente trabajo cumplió con el objetivo de obtener resultados fiables mediante la aplicación de cálculos de inferencia probabilista para la ocurrencia de sismos de magnitud significativa en la costa del Pacífico mexicano. A través de la implementación sistemática de la metodología propuesta, se lograron identificar patrones estadísticamente significativos en el comportamiento sísmico de las regiones analizadas.

Los resultados obtenidos permiten establecer que el análisis diferenciado por regiones es fundamental para la comprensión del fenómeno sísmico en México. Se demostró que Chiapas presenta la mayor cantidad de registros sísmicos (629 eventos $\geq 5^\circ$), mientras que Michoacán, con menor cantidad de eventos (90), registra la mayor dispersión en magnitudes, incluyendo el sismo histórico de 1985. Estas diferencias regionales confirman que no es apropiado aplicar un modelo único para toda la costa del Pacífico mexicano ni para todo el país.

Un hallazgo muy relevante fue que ninguna de las 20 distribuciones probadas se ajustó satisfactoriamente los datos de Sismos Totales, lo cual indica la complejidad del fenómeno sísmico cuando se utilizan todos los eventos registrados, indicando que tener más datos no es garantía de una mejor representación del comportamiento sísmico. Sin embargo, para las Magnitudes Máximas Anuales se identificaron distribuciones específicas que sí ajustaron (p-valor $> 0.05$): Gumbel para Oaxaca, Weibull para Guerrero y Chiapas, GEV para Michoacán, Logística para Resto Nacional y Normal para Sismos Nacionales. Este resultado valida el enfoque de valores extremos para la inferencia sísmica.

Las pruebas de hipótesis revelaron diferencias estadísticamente significativas entre las regiones en términos de media, varianza y proporción de sismos mayores al umbral crítico de 6.5°. Particularmente relevante es que Guerrero presenta la mayor proporción de sismos fuertes (17.58\% para Sismos Totales), mientras que a nivel nacional esta proporción es del 8.84\%, información muy importante para fortalecer la cultura de la prevención.

El análisis temporal identificó a septiembre como el mes de mayor actividad sísmica tanto para Sismos Totales como para Magnitudes Máximas a nivel nacional, coincidiendo con eventos históricos significativos (sismos de 1985 y 2017).

La estimación del tamaño mínimo de muestra demostró que los datos históricos disponibles son suficientes para realizar inferencias confiables. Por ejemplo, Chiapas requiere 483 muestras para estimar la media con un error de $\pm 0.05$, y cuenta con 629 registros, validando la robustez estadística de los análisis realizados.

Los cálculos de inferencia probabilística mediante teoría de valores extremos proporcionaron estimaciones cuantitativas del riesgo sísmico en cada región. Los periodos de retorno calculados para horizontes de 10, 20, 50 y 100 años revelan que Guerrero presenta los niveles de retorno más altos, con magnitudes esperadas superiores a 7.5° para periodos de 50 años, mientras que otras regiones muestran niveles entre 6.8° y 7.2° para el mismo horizonte temporal. La implementación de bootstrap paramétrico con 10,000 iteraciones \cite{papalexiou2020random} permitió cuantificar la incertidumbre asociada a estas estimaciones mediante intervalos de confianza al 95\%, demostrando que incluso considerando la variabilidad inherente, las magnitudes esperadas en la costa del Pacífico mexicano representan un peligro significativo. El análisis de probabilidades de excedencia mostró que la probabilidad de experimentar un sismo con magnitud $\geq 7.0°$ en los próximos 10 años varía considerablemente entre regiones: Guerrero presenta una probabilidad del 42\%, Oaxaca del 35\%, Michoacán del 28\% y Chiapas del 31\%, mientras que a nivel nacional esta probabilidad alcanza el 68\%. El cálculo del índice de proximidad temporal reveló que varias regiones han excedido el tiempo medio de recurrencia ($\text{IPT} > 1.0$), lo que sugiere una acumulación de energía y un incremento en la probabilidad de ocurrencia de eventos significativos en el corto plazo. Estos resultados proporcionan información cuantitativa valiosa para la toma de decisiones en materia de prevención y mitigación del riesgo sísmico.

La comparación de la metodología implementada con los enfoques revisados en el estado del arte revela tanto convergencias como aportaciones distintivas de este trabajo. Al igual que los estudios de \citeasnoun{mignan2021best} y \citeasnoun{yaghmaeisakbegh2022regional}, este trabajo adoptó el uso de magnitudes máximas anuales y distribuciones de valores extremos, validando empíricamente la recomendación de \citeasnoun{bommier2023peak} sobre la superioridad del método de máximos anuales para conjuntos de datos con menos de 150 años de registros. Sin embargo, a diferencia de trabajos previos que se enfocan en una sola región o utilizan una distribución única, esta investigación implementó un análisis comparativo sistemático probando 20 distribuciones probabilísticas diferentes para cada región, permitiendo identificar que cada estado de la costa del Pacífico mexicano requiere un modelo probabilístico específico. Esta diferenciación regional no había sido documentada con este nivel de detalle en estudios previos sobre México. La integración del análisis temporal mediante el índice de proximidad temporal y el coeficiente de variación de recurrencia, inspirada en los trabajos de \citeasnoun{ramirezgaytan2021earthquake} y \citeasnoun{sanchezsilva2020real}, complementa las estimaciones puramente probabilísticas con consideraciones sobre la memoria temporal del proceso sísmico, superando las limitaciones de los modelos de Poisson tradicionales. Mientras que otros enfoques revisados se inclinan hacia técnicas de inteligencia artificial y aprendizaje profundo \cite{abebe2023earthquakes,jena2021earthquake}, este trabajo demuestra que los métodos estadísticos clásicos robustos, cuando se aplican con rigor metodológico y se adaptan al contexto regional específico, pueden proporcionar estimaciones confiables y más interpretables para la toma de decisiones. La construcción del índice compuesto de peligrosidad sísmica regional representa una síntesis metodológica que integra elementos de múltiples estudios previos \cite{convertito2020combining,zuniga2022first}, adaptándolos al contexto mexicano y proporcionando una herramienta práctica para la comparación objetiva del riesgo entre regiones.