\chapter{Conclusiones \label{cap:Conclusiones}}

\noindent
El presente trabajo cumplió con el objetivo de obtener resultados fiables mediante la aplicación de cálculos de inferencia probabilista para la ocurrencia de sismos de magnitud significativa en la costa del Pacífico mexicano. A través de la implementación sistemática de la metodología propuesta, se lograron identificar patrones estadísticamente significativos en el comportamiento sísmico de las regiones analizadas.

Los resultados obtenidos permiten establecer que el análisis diferenciado por regiones es fundamental para la comprensión del fenómeno sísmico en México. Se demostró que Chiapas presenta la mayor cantidad de registros sísmicos (629 eventos \( \geq 5^\circ \)), mientras que Michoacán, con menor cantidad de eventos (90), registra la mayor dispersión en magnitudes, incluyendo el sismo histórico de 1985. Estas diferencias regionales confirman que no es apropiado aplicar un modelo único para toda la costa del Pacífico mexicano ni para todo el país.

Un hallazgo muy relevante fue que ninguna de las 20 distribuciones probadas se ajustó satisfactoriamente los datos de Sismos Totales, lo cual indica la complejidad del fenómeno sísmico cuando se utilizan todos los eventos registrados, indicando que tener más datos no es garantía de una mejor representación del comportamiento sísmico. Sin embargo, para las Magnitudes Máximas Anuales se identificaron distribuciones específicas que se si (p-valor \( > 0.05 \)): Gumbel para Oaxaca, Weibull para Guerrero y Chiapas, GEV para Michoacán, Logística para Resto Nacional y Normal para Sismos Nacionales. Este resultado valida el enfoque de valores extremos para la inferencia sísmica.

Las pruebas de hipótesis revelaron diferencias estadísticamente significativas entre las regiones en términos de media, varianza y proporción de sismos mayores al umbral crítico de 6.5°. Particularmente relevante es que Guerrero presenta la mayor proporción de sismos fuertes (17.58\% para Sismos Totales), mientras que a nivel nacional esta proporción es del 8.84\%, información muy importante para fortalecer la cultura de la prevención.

El análisis temporal identificó a septiembre como el mes de mayor actividad sísmica tanto para Sismos Totales como para Magnitudes Máximas a nivel nacional, coincidiendo con eventos históricos significativos (sismos de 1985 y 2017).

La estimación del tamaño mínimo de muestra demostró que los datos históricos disponibles son suficientes para realizar inferencias confiables. Por ejemplo, Chiapas requiere 483 muestras para estimar la media con un error de ±0.05, y cuenta con 629 registros, validando la robustez estadística de los análisis realizados.