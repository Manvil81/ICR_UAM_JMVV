\chapter{Marco teórico \label{cap:MarcoTeorico}}

\noindent
Se procede a la revisión de artículos y publicaciones que tienen que ver directamente con el tema de la problemática a solucionar.

A continuación, se presenta la Tabla \ref{tab:articulos_seleccionados} que muestra una selección de los artículos relacionados a la metodología que se utiliza a lo largo de este trabajo.

\small
\begin{longtable}{|c|p{14cm}|}
\caption{Artículos seleccionados.} \label{tab:articulos_seleccionados} \\
\hline
\rowcolor{gray!60}
\textbf{Año} & \makebox[14cm][c]{\textbf{Título y Autor}} \\
\hline
\endfirsthead

\hline
\rowcolor{gray!60}
\textbf{Año} & \makebox[14cm][c]{\textbf{Título y Autor}} \\
\hline
\endhead

\hline
\endfoot

\hline
\endlastfoot

\rowcolor{gray!20}
2023 & ``Earthquakes magnitude prediction using deep learning for the Horn of Africa'' \cite{abebe2023earthquakes} \\
\hline
2020 & ``Application of Artificial Intelligence in Predicting Earthquakes: State-of-the-Art and Future Challenges'' \cite{albanna2020application} \\
\hline
\rowcolor{gray!20}
2007 & ``Análisis geográfico y estadístico de la sismicidad en la costa mexicana del Pacífico'' \cite{barrientos2007analisis} \\
\hline
2021 & ``Time-Dependent Seismic Hazard Analysis for Induced Seismicity: The Case of St Gallen (Switzerland), Geothermal Field'' \cite{convertito2021time} \\
\hline
\rowcolor{gray!20}
2005 & ``A probabilistic prediction of the next strong earthquake in the Acapulco-San Marcos segment, Mexico'' \cite{ferraes2005probabilistic} \\
\hline
2022 & ``Long-Term Forecasting of Strong Earthquakes in North America, South America, Japan, Southern China and Northern India With Machine Learning'' \cite{velascoherrera2022long} \\
\hline
\rowcolor{gray!20}
2021 & ``Earthquake risk assessment in NE India using deep learning and geospatial analysis'' \cite{jena2021earthquake} \\
\hline
2023 & ``Fundamental study on probabilistic generative modeling of earthquake ground motion time histories using generative adversarial networks'' \cite{matsumoto2023fundamental} \\
\hline
\rowcolor{gray!20}
2020 & ``Ground motion prediction equation for crustal earthquakes in Taiwan'' \cite{phung2020ground} \\
\hline
1935 & ``An instrumental earthquake magnitude scale'' \cite{richter1935instrumental} \\
\hline
\rowcolor{gray!20}
2021 & ``Theoretical methodological aspects about earthquake prediction'' \cite{galbanrodriguez2021theoretical} \\
\hline
2023 & ``Probabilistic seismic hazard assessment for Western Mexico'' \cite{sawires2023probabilistic} \\
\hline
\rowcolor{gray!20}
2025 & ``Hypothesis Testing, P Values, Confidence Intervals, and Significance'' \cite{shreffler2025hypothesis} \\
\hline
2024 & ``Comparative analysis of continuous probability distributions for modeling maximum flood levels'' \cite{shobanke2024comparative} \\
\hline
\rowcolor{gray!20}
2021 & ``Ground motion prediction equation for earthquakes along the Western Himalayan arc'' \cite{singh2021ground} \\
\hline
2019 & ``Descriptive analysis and earthquake prediction using boxplot interpretation of soil radon time series data'' \cite{tareen2019descriptive} \\
\hline
\end{longtable}
%% Agregar el nombre de la tabla a la segunda parte que se dividió en otra hoja
% \vspace{0.5em}
% \noident\textbf{Tabla \ref{tab:articulos_seleccionados}: Artículos seleccionados.}

\section{Síntesis de los artículos}

\noindent
Ahora se realiza una descripción del contenido de los artículos seleccionados.

En el artículo de \citeasnoun{abebe2023earthquakes} se expone que los terremotos son vibraciones de la superficie de la Tierra que pueden causar temblores, incendios, deslizamientos de tierra y fisuras. Por lo cual, en esta investigación se aplicó una técnica basada en el aprendizaje profundo, un algoritmo transformador que tiene como fin predecir las magnitudes de los terremotos, utilizando los datos disponibles para el Cuerno de África.

En el artículo de \citeasnoun{albanna2020application} se menciona que las técnicas basadas en la inteligencia artificial producen un resultado prometedor en la predicción de terremotos, gracias a su capacidad de encontrar patrones ocultos en los datos. Por lo tanto, en este trabajo se exploran las fechas de predicción para terremotos que fueron arrojadas mediante la utilización de técnicas basadas en la inteligencia artificial.

Para el artículo de \citeasnoun{barrientos2007analisis} se efectúa el análisis estadístico y geográfico para eventos de tipo sísmico que se han registrado en la costa del Pacífico mexicano en el siglo XX, describiendo las ocurrencias de estos como un proceso de puntos y proponiendo el uso de modelos lineales de tipo log-lineal para ajustar sus tasas de ocurrencia.

\citeasnoun{convertito2021time} presenta una técnica para modificar el análisis probabilístico estándar de la peligrosidad sísmica aplicado a la sismicidad inducida, el cual se apoya en el análisis probabilista de peligrosidad sísmica, particularmente las profundidades relativamente bajas, la pequeña magnitud, la correlación con las operaciones de campo y el tiempo de recurrencia no Poisson. Esta técnica permite utilizar modelos no Poisson (Brownian Passage Time, Weibull, gamma y Epidemic Type Aftershock Sequence) en los cuales, sus parámetros son ajustados al registro de sismicidad que se implementa en las etapas de las operaciones de campo.

En el artículo \citeasnoun{ferraes2005probabilistic} se supone una distribución gamma y una distribución lognormal para los intervalos de tiempo de recurrencia de grandes terremotos. Se implementa el uso de las probabilidades condicionales de recurrencia para representar de forma válida y razonable la posible ocurrencia de grandes sismos.

Para el artículo \citeasnoun{velascoherrera2022long} se menciona que la capacidad para pronosticar terremotos fuertes a largo plazo es esencial para de esta forma minimizar los riesgos y las vulnerabilidades de las personas que viven en áreas consideradas altamente sísmicas. Fueron analizados diferentes patrones sísmicos en el territorio de Japón para de esta forma crear un modelo probabilista de inferencia sísmica a largo plazo para cada zona sísmica, utilizando el método de aprendizaje automático bayesiano.

\citeasnoun{jena2021earthquake} expone un modelo de mapeo de riesgo sísmico basado en el aprendizaje profundo. Desarrollando un modelo de red neuronal convolucional con el propósito de evaluar la probabilidad de terremotos. Asimismo, se lleva a cabo la vulnerabilidad utilizando el proceso de jerarquía analítica, la teoría de intersección de Venn para el peligro y finalmente el modelo integrado para el mapeo de riesgos.

En el estudio del artículo de \citeasnoun{matsumoto2023fundamental} se propone un modelo probabilista para la inferencia de sismos y terremotos, el cual es denominado como modelo de generación de movimiento de suelo, con el propósito de crear datos de la historia del tiempo de movimiento del suelo directamente. Asimismo, se plantea un método para evaluar de forma cuantitativa y cualitativa el rendimiento del modelo construido y se optimiza el modelo de generación de movimiento del suelo para lograr un alto rendimiento desde las perspectivas de la ingeniería sísmica y el aprendizaje profundo.

Para el artículo de \citeasnoun{phung2020ground} se desarrolló una ecuación de predicción del movimiento del suelo con el fin de estimar las amplitudes horizontales del movimiento del suelo causadas por terremotos de la corteza. Este fue basado en un conjunto de datos que incluye los terremotos ocurridos en Taiwán.

En el artículo de \citeasnoun{richter1935instrumental} se introduce la escala de magnitud Richter para los terremotos, la cual se propone como herramienta para medir la magnitud de un sismo. Esta escala utiliza amplitudes de ondas sísmicas registradas por los sismógrafos durante un terremoto y proporciona una medida cuantitativa de la energía liberada por el mismo. Asimismo se habla acerca de cómo calibrar la escala y de su trabajo para desarrollarla analizando eventos sísmicos en el sur de California.

Para el artículo de \citeasnoun{galbanrodriguez2021theoretical} se realiza un recorrido por el tema de los terremotos, partiendo de su marco conceptual, se revisan algunas de las principales teorías de su predicción sísmica, con el fin de que científicos puedan aplicar los preceptos metodológicos expuestos a su trabajo diario.

Para el artículo de \citeasnoun{sawires2023probabilistic}, se lleva a cabo un análisis probabilístico actualizado del peligro sísmico para el occidente de México. Este estudio utiliza un catálogo de terremotos unificado y actualizado, así como modelos de fuentes sísmicas, para evaluar el peligro sísmico en términos de aceleración máxima del terreno y aceleración espectral, considerando la incertidumbre en la estimación de parámetros sismológicos clave.

En el artículo de \citeasnoun{shreffler2025hypothesis}, se presentan los conceptos fundamentales de las pruebas de hipótesis, los valores p, los intervalos de confianza y la significación estadística. El trabajo se enfoca en cómo estos elementos son cruciales en la investigación clínica para guiar la toma de decisiones basada en la evidencia, y explica su interrelación, incluyendo la relevancia del tamaño de la muestra para la interpretación de los resultados.

Para el artículo de \citeasnoun{shobanke2024comparative}, se realiza un análisis comparativo del rendimiento de varias distribuciones de probabilidad continuas, incluyendo la Normal, Cauchy, Chi-Cuadrada, Normal Estándar y t de Student. El objetivo es modelar los niveles máximos de inundación, utilizando criterios de selección de modelos como el Criterio de Información de Akaike para identificar la distribución que mejor se ajusta a los datos.

Según \citeasnoun{singh2021ground}, un elemento crítico para la estimación del peligro sísmico es la ecuación de predicción del movimiento del suelo, el cual logra relacionar la intensidad sísmica esperada en un punto de un terremoto de una magnitud y una ubicación determinada. En este estudio se utilizaron un conjunto de datos de terremotos, así como de réplicas para derivar la ecuación de predicción del movimiento del suelo para terremotos a lo largo del arco del Himalaya y proporcionar una estimación fiable de los parámetros del movimiento del suelo en ciertos sitios a lo largo del arco y de la península.

\citeasnoun{tareen2019descriptive} presenta un análisis estadístico descriptivo y una predicción de terremotos contingentes basada en datos de series temporales de radón en el suelo. Los datos fueron recopilados durante una línea de falla la cual pasa por debajo de Muzaffarabad, durante el periodo de un año. El análisis de series temporales de radón mediante diagramas de caja y parámetros meteorológicos muestran patrones específicos en las concentraciones de radón, esto debido a las actividades sísmicas subterráneas previas al terremoto.