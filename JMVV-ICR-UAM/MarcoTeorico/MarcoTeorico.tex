\chapter{Marco teórico \label{cap:MarcoTeorico}}

\noindent
Se procede a la revisión de artículos y publicaciones que tienen que ver directamente con el tema de la problemática a solucionar. 

A continuación, se presenta la Tabla \ref{tab:articulos_seleccionados} que muestra una selección de los artículos relacionados a la metodología que se utiliza a lo largo de este trabajo.

%% \small
\begingroup
\footnotesize
\begin{longtable}{|c|p{14cm}|}
\caption{Artículos seleccionados. } \label{tab:articulos_seleccionados} \\
\hline
\rowcolor{gray! 60}
\textbf{Año} & \makebox[14cm][c]{\textbf{Título y Autor}} \\
\hline
\endfirsthead
\hline
\rowcolor{gray!60}
\textbf{Año} & \makebox[14cm][c]{\textbf{Título y Autor}} \\
\hline
\endhead
\hline
\endfoot
\hline
\endlastfoot
\rowcolor{gray!20}
2023 & ``Earthquakes magnitude prediction using deep learning for the Horn of Africa'' \cite{abebe2023earthquakes} \\
\hline
2020 & ``Application of Artificial Intelligence in Predicting Earthquakes: State-of-the-Art and Future Challenges'' \cite{albanna2020application} \\
\hline
\rowcolor{gray! 20}
2007 & ``Análisis geográfico y estadístico de la sismicidad en la costa mexicana del Pacífico'' \cite{barrientos2007analisis} \\
\hline
2023 & ``Peak-over-threshold versus annual maxima:  Which approach is better for extreme value analysis?'' \cite{bommier2023peak} \\
\hline
\rowcolor{gray! 20}
2001 & ``An Introduction to Statistical Modeling of Extreme Values'' \cite{coles2001introduction} \\
\hline
2021 & ``Extreme Value Theory - 20 years on'' \cite{coles2021extreme} \\
\hline
\rowcolor{gray!20}
2020 & ``Combining stress transfer and source clustering to forecast seismicity'' \cite{convertito2020combining} \\
\hline
2021 & ``Time-Dependent Seismic Hazard Analysis for Induced Seismicity:  The Case of St Gallen (Switzerland), Geothermal Field'' \cite{convertito2021time} \\
\hline
\rowcolor{gray!20}
1968 & ``Engineering seismic risk analysis'' \cite{cornell1968engineering} \\
\hline
2005 & ``A probabilistic prediction of the next strong earthquake in the Acapulco-San Marcos segment, Mexico'' \cite{ferraes2005probabilistic} \\
\hline
\rowcolor{gray!20}
2022 & ``Long-Term Forecasting of Strong Earthquakes in North America, South America, Japan, Southern China and Northern India With Machine Learning'' \cite{velascoherrera2022long} \\
\hline
2021 & ``Earthquake risk assessment in NE India using deep learning and geospatial analysis'' \cite{jena2021earthquake} \\
\hline
\rowcolor{gray!20}
1995 & ``Continuous Univariate Distributions'' \cite{johnson1995continuous} \\
\hline
2023 & ``Fundamental study on probabilistic generative modeling of earthquake ground motion time histories using generative adversarial networks'' \cite{matsumoto2023fundamental} \\
\hline
\rowcolor{gray!20}
2021 & ``Best practices in physics-based fault rupture forecasting for seismic hazard assessment of nuclear installations:  issues and challenges towards full integration'' \cite{mignan2021best} \\
\hline
2020 & ``Ground motion prediction equation for crustal earthquakes in Taiwan'' \cite{phung2020ground} \\
\hline
\rowcolor{gray!20}
2021 & ``Earthquake source parameters of the Michoacán seismic gap'' \cite{ramirezgaytan2021earthquake} \\
\hline
1935 & ``An instrumental earthquake magnitude scale'' \cite{richter1935instrumental} \\
\hline
\rowcolor{gray!20}
2021 & ``Theoretical methodological aspects about earthquake prediction'' \cite{galbanrodriguez2021theoretical} \\
\hline
2020 & ``Real-time updating of the seismic risk of interdependent infrastructure systems using Bayesian networks'' \cite{sanchezsilva2020real} \\
\hline
\rowcolor{gray!20}
2023 & ``Probabilistic seismic hazard assessment for Western Mexico'' \cite{sawires2023probabilistic} \\
\hline
2021 & ``Understanding persistence to avoid underestimation of collective flood risk'' \cite{serinaldi2021understanding} \\
\hline
\rowcolor{gray!20}
2025 & ``Hypothesis Testing, P Values, Confidence Intervals, and Significance'' \cite{shreffler2025hypothesis} \\
\hline
2024 & ``Comparative analysis of continuous probability distributions for modeling maximum flood levels'' \cite{shobanke2024comparative} \\
\hline
\rowcolor{gray!20}
2020 & ``Deadly intraslab Mexico earthquake of 19 September 2017 (Mw 7.1): Ground motion and damage pattern in Mexico City'' \cite{singh2020deadly} \\
\hline
2021 & ``Ground motion prediction equation for earthquakes along the Western Himalayan arc'' \cite{singh2021ground} \\
\hline
\rowcolor{gray!20}
2019 & ``Descriptive analysis and earthquake prediction using boxplot interpretation of soil radon time series data'' \cite{tareen2019descriptive} \\
\hline
2022 & ``Regional probability distribution of ground motion parameters using machine learning and Bayesian approaches'' \cite{yaghmaeisakbegh2022regional} \\
\hline
\rowcolor{gray!20}
2022 & ``A first-order seismotectonic regionalization of Mexico for seismic hazard and risk estimation'' \cite{zuniga2022first} \\
\hline
\end{longtable}
\endgroup



\section{Síntesis de los artículos}

\noindent
Ahora se realiza una descripción del contenido de los artículos seleccionados. 

En el artículo de \citeasnoun{abebe2023earthquakes} se expone que los terremotos son vibraciones de la superficie de la Tierra que pueden causar temblores, incendios, deslizamientos de tierra y fisuras en el terreno que representan una amenaza significativa para la vida humana y la infraestructura.  Los autores desarrollan un modelo de aprendizaje profundo para la predicción de magnitudes sísmicas en el Cuerno de África, utilizando datos históricos del catálogo sísmico de la región.  El modelo propuesto emplea redes neuronales recurrentes de tipo LSTM (Long Short-Term Memory) para capturar patrones temporales en las secuencias sísmicas.  Los resultados muestran que el modelo alcanza una precisión significativa en la predicción de magnitudes, superando a métodos estadísticos tradicionales en la región de estudio. 

En el artículo de \citeasnoun{albanna2020application} se presenta una revisión del estado del arte relacionado a la aplicación de técnicas de inteligencia artificial (IA) en la predicción de sismos. Los autores analizan diversos enfoques que incluyen redes neuronales artificiales, máquinas de soporte vectorial, algoritmos genéticos y sistemas difusos. El artículo identifica que las técnicas de IA han mostrado resultados prometedores en la identificación de patrones precursores de sismos, aunque señala que la predicción sísmica sigue siendo un desafío abierto debido a la complejidad inherente de los procesos tectónicos. Los autores concluyen que la integración de múltiples técnicas de IA con datos geofísicos diversos representa una dirección prometedora para futuras investigaciones.

Para el artículo de \citeasnoun{barrientos2007analisis} se realiza el análisis estadístico y geográfico para eventos de tipo sísmico que se han registrado en la costa del Pacífico mexicano en el periodo de 1990 a 2004. El estudio utiliza datos del Servicio Sismológico Nacional para caracterizar la distribución espacial y temporal de la sismicidad en la región. Los autores aplican técnicas de estadística descriptiva y análisis de frecuencias para identificar patrones en la ocurrencia de sismos. Los resultados muestran que la actividad sísmica en la costa del Pacífico mexicano presenta variaciones significativas entre diferentes segmentos de la zona de subducción, con algunas áreas mostrando mayor frecuencia de eventos de magnitud moderada a alta. 

En el artículo de \citeasnoun{bommier2023peak} se realiza un análisis comparativo entre dos metodologías ampliamente utilizadas en el análisis de valores extremos: el enfoque de excedencias sobre umbral (peak-over-threshold, POT) y el método de máximos anuales (block maxima). El autor evalúa ambos enfoques utilizando datos sintéticos y reales, analizando su desempeño en términos de sesgo, varianza y eficiencia en la estimación de cuantiles extremos. Los resultados indican que el método de máximos anuales es preferible cuando se dispone de series temporales largas y los eventos extremos son relativamente frecuentes, mientras que el enfoque POT puede ser más eficiente con muestras pequeñas pero requiere una selección cuidadosa del umbral.

\citeasnoun{coles2001introduction} presenta en su libro ``An Introduction to Statistical Modeling of Extreme Values'' el marco teórico fundamental de la teoría de valores extremos (Extreme Value Theory, EVT). El autor desarrolla de manera rigurosa los fundamentos matemáticos de las distribuciones de valores extremos, incluyendo la distribución generalizada de valores extremos (GEV), la distribución de Gumbel, la distribución de Weibull y la distribución de Fréchet. El libro presenta métodos de estimación de parámetros, construcción de intervalos de confianza y cálculo de niveles de retorno. Esta obra constituye una referencia fundamental para el análisis probabilístico de eventos extremos en diversas disciplinas, incluyendo la sismología, hidrología y ciencias atmosféricas.

\citeasnoun{coles2021extreme} publican una revisión actualizada del estado del arte en teoría de valores extremos, veinte años después del libro seminal de \citeasnoun{coles2001introduction}. Los autores examinan los avances metodológicos más significativos en el campo, incluyendo nuevos métodos de estimación, técnicas para datos no estacionarios y enfoques bayesianos. El artículo discute las aplicaciones emergentes de la teoría de valores extremos en áreas como el cambio climático, riesgo financiero y peligrosidad sísmica. Los autores identifican desafíos abiertos y direcciones futuras de investigación, destacando la necesidad de desarrollar métodos que permitan modelar la dependencia espacial y temporal en eventos extremos.

En otro trabajo, \citeasnoun{convertito2020combining} desarrollan un modelo integrado que combina la transferencia de esfuerzos tectónicos con el análisis de agrupamiento espacial de sismos para mejorar el pronóstico de la sismicidad.  El modelo propuesto permite estimar cambios en la tasa de sismicidad considerando tanto los efectos de la transferencia de esfuerzos producida por sismos principales como los patrones de agrupamiento espacial observados en las réplicas.  Los autores aplican el modelo a secuencias sísmicas en Italia, demostrando que la integración de ambos enfoques mejora significativamente la capacidad predictiva respecto a modelos que consideran cada factor por separado.

\citeasnoun{convertito2021time} presentan una técnica innovadora para evaluar el riesgo de sismos inducidos por proyectos de actividad geotérmica.  La técnica modifica el enfoque tradicional de análisis probabilístico de peligrosidad sísmica (PSHA) para incorporar la variabilidad temporal de la sismicidad inducida.  Los autores aplican la metodología al campo geotérmico de St.  Gallen en Suiza, donde la inyección de fluidos ha generado actividad sísmica.  Los resultados demuestran que el enfoque dependiente del tiempo proporciona estimaciones más realistas del peligro sísmico durante las diferentes fases de operación del proyecto geotérmico. 

\citeasnoun{cornell1968engineering} presenta el artículo fundacional del análisis probabilístico de peligrosidad sísmica (Probabilistic Seismic Hazard Analysis, PSHA). El autor desarrolla el marco metodológico para cuantificar la probabilidad de que se excedan diferentes niveles de intensidad sísmica en un sitio determinado durante un periodo de tiempo específico. El artículo introduce el concepto de periodo de retorno y establece las bases matemáticas para calcular probabilidades de excedencia asumiendo independencia temporal entre eventos sísmicos. Esta metodología se ha convertido en el estándar internacional para la evaluación de peligrosidad sísmica y es ampliamente utilizada en el diseño de infraestructura crítica, normativas de construcción y políticas de gestión del riesgo sísmico.

\citeasnoun{ferraes2005probabilistic} desarrolla un análisis probabilista para estimar la ocurrencia del próximo gran terremoto en el segmento Acapulco-San Marcos, México.  Estudia los intervalos de tiempos intereventos sísmicos y encuentra que la distribución lognormal describe adecuadamente la recurrencia de terremotos grandes en esta región. El autor utiliza datos históricos de sismos significativos ocurridos en el segmento desde el siglo XIX para calibrar el modelo probabilístico. Los resultados proporcionan estimaciones de la probabilidad condicional de ocurrencia de un sismo mayor en diferentes horizontes temporales, información útil para la planificación de medidas de prevención y mitigación del riesgo sísmico.

Para el artículo de \citeasnoun{velascoherrera2022long} se señala la importancia de pronosticar terremotos fuertes a largo plazo como una herramienta fundamental para minimizar los riesgos y las vulnerabilidades de las comunidades afectadas por la actividad sísmica.  Los autores desarrollan un modelo de aprendizaje automático que utiliza datos históricos de sismos para identificar patrones de recurrencia en diferentes regiones del mundo, incluyendo América del Norte, América del Sur, Japón, el sur de China y el norte de la India. El modelo propuesto logra identificar ciclos de actividad sísmica que permiten estimar ventanas temporales con mayor probabilidad de ocurrencia de sismos significativos.

\citeasnoun{jena2021earthquake} desarrollan un modelo integrado para la evaluación del riesgo sísmico en el noroeste de la India, combinando técnicas de aprendizaje profundo y análisis geoespacial. El estudio utiliza datos sísmicos históricos, información geológica y parámetros geomorfológicos para construir mapas de susceptibilidad sísmica. Los autores aplican redes neuronales convolucionales para identificar patrones espaciales asociados con la ocurrencia de sismos.  Los resultados muestran que el modelo híbrido propuesto supera en precisión a los métodos tradicionales de evaluación del riesgo sísmico en la región de estudio.

\citeasnoun{johnson1995continuous} presentan en su libro ``Continuous Univariate Distributions'' una referencia exhaustiva sobre distribuciones de probabilidad continuas. Los autores desarrollan de manera rigurosa las propiedades matemáticas, métodos de estimación de parámetros y aplicaciones de una amplia variedad de distribuciones, incluyendo la distribución normal, lognormal, Weibull, Gumbel, logística y la distribución generalizada de valores extremos.  El libro proporciona fórmulas para el cálculo de cuantiles, momentos y funciones generadoras de momentos.  Esta obra constituye una referencia fundamental para la selección y aplicación de distribuciones de probabilidad en el análisis de datos, incluyendo el modelado de valores extremos como magnitudes sísmicas máximas.

En el artículo de \citeasnoun{matsumoto2023fundamental} se propone un modelo probabilista para la predicción del movimiento del suelo por terremotos llamado modelo de generación de movimiento del suelo.  El estudio utiliza redes generativas adversarias (GANs) para generar historias temporales sintéticas del movimiento del suelo que preservan las características estadísticas de los registros sísmicos reales. Los autores entrenan el modelo con datos de acelerogramas registrados en Japón, demostrando que las señales generadas reproducen adecuadamente el contenido frecuencial y la duración de los movimientos sísmicos observados. 

\citeasnoun{mignan2021best} desarrollan un marco metodológico integral para el pronóstico probabilístico de rupturas sísmicas aplicado específicamente a la evaluación de peligrosidad sísmica en instalaciones nucleares. Los autores revisan las mejores prácticas en modelado de fuentes sísmicas, incluyendo la caracterización de fallas activas y el cálculo de tasas de ocurrencia. El artículo discute los desafíos asociados con la integración de modelos físicos de ruptura con el análisis probabilístico tradicional, proponiendo enfoques híbridos que combinan información geológica, geodésica y sismológica para mejorar las estimaciones de peligrosidad. 

\citeasnoun{phung2020ground} desarrollan una ecuación de predicción del movimiento del suelo (GMPE) específica para sismos corticales en el territorio de Taiwán. El modelo propuesto utiliza un amplio conjunto de registros sísmicos para calibrar los coeficientes de la ecuación, considerando efectos de la magnitud, distancia, tipo de falla y condiciones locales del suelo. Los autores validan el modelo comparando las predicciones con observaciones independientes, demostrando que la GMPE específica para Taiwán proporciona estimaciones más precisas que las ecuaciones genéricas desarrolladas para otras regiones. 

\citeasnoun{ramirezgaytan2021earthquake} realizan un análisis probabilístico exhaustivo de la brecha sísmica de Michoacán utilizando datos del Servicio Sismológico Nacional (SSN) que abarcan desde 1900 hasta la actualidad. Los autores analizan los parámetros de fuente de los sismos ocurridos en la región, identificando patrones de recurrencia y estimando el potencial sísmico de la brecha. El estudio proporciona información relevante para la evaluación de la peligrosidad sísmica en una de las regiones con mayor riesgo de México, donde se espera la ocurrencia de un sismo de gran magnitud en las próximas décadas.

En el artículo de \citeasnoun{richter1935instrumental} se introduce la escala de magnitud Richter para los terremotos, la cual se propone como herramienta para medir la magnitud de los sismos. Esta escala logarítmica permite cuantificar la energía liberada por un terremoto a partir de la amplitud máxima registrada en un sismógrafo estándar. El autor establece las bases para la comparación objetiva entre sismos ocurridos en diferentes lugares y tiempos, contribuyendo significativamente al desarrollo de la sismología cuantitativa moderna.

En el artículo de \citeasnoun{galbanrodriguez2021theoretical} se realiza un análisis teórico y metodológico sobre la predicción de terremotos, partiendo de su marco conceptual que contextualiza el fenómeno sísmico dentro de las ciencias de la Tierra. El autor revisa las diferentes aproximaciones a la predicción sísmica, desde los métodos empíricos basados en precursores hasta los modelos estadísticos y físicos. El artículo discute las limitaciones inherentes a la predicción sísmica y propone criterios para evaluar la validez de los pronósticos, contribuyendo a establecer un marco metodológico riguroso para la investigación en este campo.

\citeasnoun{sanchezsilva2020real} implementan un modelo de renovación para la actualización en tiempo real del riesgo sísmico en sistemas de infraestructura interdependiente, con aplicaciones a datos sísmicos de la costa del Pacífico mexicano. Los autores desarrollan un enfoque basado en redes bayesianas que permite incorporar nueva información sísmica conforme se registran eventos, actualizando dinámicamente las estimaciones de riesgo. El modelo considera las interdependencias entre diferentes componentes de infraestructura, proporcionando una herramienta para la gestión del riesgo en sistemas complejos.

Para el artículo de \citeasnoun{sawires2023probabilistic} se lleva a cabo una evaluación probabilista actualizada del peligro sísmico en el occidente de México.  El estudio utiliza un catálogo sísmico actualizado y modelos de fuentes sísmicas mejorados para calcular curvas de peligrosidad y mapas de aceleración esperada para diferentes periodos de retorno.  Los autores comparan sus resultados con estudios previos, identificando diferencias significativas en algunas regiones que se atribuyen a la incorporación de nuevas fuentes sísmicas y a la utilización de ecuaciones de predicción del movimiento del suelo más recientes.

\citeasnoun{serinaldi2021understanding} desarrollan una metodología para el cálculo de probabilidades de excedencia que considera explícitamente la autocorrelación temporal en series de valores extremos. Los autores demuestran que ignorar la persistencia temporal puede conducir a subestimaciones significativas del riesgo colectivo de eventos extremos. El artículo proporciona herramientas para cuantificar el efecto de la autocorrelación en las estimaciones de periodo de retorno y probabilidad de excedencia, con aplicaciones a datos hidrológicos que pueden extenderse al análisis de secuencias sísmicas.

En el artículo de \citeasnoun{shreffler2025hypothesis} se presenta una revisión de los principios estadísticos esenciales para la investigación clínica, incluyendo pruebas de hipótesis, valores p, intervalos de confianza y significancia estadística.  Aunque el enfoque del artículo es clínico, los conceptos presentados son fundamentales para cualquier análisis estadístico riguroso, incluyendo el análisis de datos sísmicos. Los autores discuten la interpretación correcta de los resultados estadísticos y las limitaciones de las pruebas de hipótesis tradicionales.

Para el artículo de \citeasnoun{shobanke2024comparative} se realiza un análisis comparativo del desempeño de varias distribuciones de probabilidad continuas en el modelado de niveles máximos de inundación.  Los autores evalúan distribuciones como la normal, lognormal, Gumbel, Weibull y la distribución generalizada de valores extremos utilizando datos hidrológicos.  El estudio emplea criterios de selección de modelos como el AIC, BIC y pruebas de bondad de ajuste para identificar la distribución óptima.  Aunque el contexto es hidrológico, la metodología es directamente aplicable al análisis de magnitudes sísmicas máximas.

\citeasnoun{singh2020deadly} realizan un análisis retrospectivo del sismo intraplaca de Puebla del 19 de septiembre de 2017 (Mw 7.1), comparando la magnitud observada y el patrón de daños en la Ciudad de México con sismos históricos similares. Los autores analizan los registros de movimiento fuerte para caracterizar las particularidades de este evento, que causó daños significativos a pesar de su distancia al epicentro. El estudio proporciona información valiosa sobre la vulnerabilidad sísmica de la Ciudad de México y la importancia de considerar sismos intraplaca en la evaluación del peligro sísmico.

En el artículo de \citeasnoun{singh2021ground} se desarrolla la ecuación de predicción del movimiento del suelo (GMPE), aplicada a los sismos ocurridos a lo largo del Arco del Himalaya occidental. Los autores utilizan datos de acelerogramas registrados en la región para calibrar los coeficientes del modelo, considerando las características tectónicas particulares de la zona de colisión continental. El estudio proporciona una herramienta para la evaluación de la peligrosidad sísmica en una de las regiones más activas del mundo. 

\citeasnoun{tareen2019descriptive} presenta un análisis estadístico descriptivo de series temporales de radón en el suelo, recopiladas durante un año en la zona de Muzaffarabad, la cual es atravesada por la falla de Jhelum que registró el sismo de Cachemira del 8 de octubre de 2005 (Mw 7.6). Los autores aplican técnicas de estadística descriptiva y análisis de boxplot para identificar anomalías en las concentraciones de radón que podrían estar asociadas con actividad sísmica.  El estudio explora el potencial del monitoreo de radón como herramienta complementaria para la identificación de precursores sísmicos.

\citeasnoun{yaghmaeisakbegh2022regional} realizan un análisis regional exhaustivo de parámetros de movimiento del suelo en Irán utilizando más de 100 años de registros sísmicos del catálogo nacional. Los autores aplican técnicas de aprendizaje automático y métodos bayesianos para desarrollar modelos de predicción del movimiento del suelo específicos para diferentes regiones tectónicas del país. El estudio demuestra que los modelos regionales proporcionan estimaciones más precisas que las ecuaciones globales, destacando la importancia de considerar las características tectónicas locales. 

\citeasnoun{zuniga2022first} proponen una regionalización sismotectónica de primer orden para México basada en el análisis estadístico exhaustivo del catálogo del Servicio Sismológico Nacional (SSN). Los autores identifican regiones con características sísmicas homogéneas utilizando técnicas de agrupamiento espacial y análisis de la distribución de magnitudes. El estudio proporciona un marco de referencia para la evaluación de la peligrosidad sísmica en México, identificando las regiones con mayor potencial de generación de sismos significativos. 