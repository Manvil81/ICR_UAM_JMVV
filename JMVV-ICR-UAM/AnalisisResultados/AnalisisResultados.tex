\chapter{Análisis de resultados \label{cap:AnalisisDeResultados}}

\noindent
El proceso de analizar y estudiar los datos sísmicos recopilados del SSN y luego aplicar la metodología descrita en el capítulo anterior nos lleva a la obtención de distintos resultados con los cuales se puede realizar la inferencia estadística de los sismos en las regiones estudiadas. A continuación, se presentan los principales resultados obtenidos después de aplicar los conocimientos descritos en el Capítulo 4 y se procede a realizar un análisis de los mismos para obtener resultados confiables basados en las tendencias de cada región.

\section{Herramientas utilizadas}
\noindent
El archivo recopilado desde el sitio del SSN es un archivo en formato csv bruto el cual contiene la información estadística de todos los sismos del país desde el 01 de enero de 1900 hasta una fecha determinada. En este caso se esta trabajando con un archivo que contiene datos sísmicos hasta el 23 de julio de 2025.

En la metodología, la Fase 2 se realiza utilizando un código desarrollado en Python para hacer el filtrado del archivo de sismos generales y para obtener los archivos individuales de cada región en formato xlsx. El código se ejecuta en Google Colab a través de servidores en la nube y el resultado son 6 archivos de Excel con la información de datos sísmicos depurada de sismos iguales o mayores a 5°.

El análisis de los archivos individuales de sismos y todos los cálculos relacionados al análisis estadístico se realizan en la herramienta R (version 4.4.2) y RStudio (2024.09.1 Build 394).

\section{Resultados de estadísticos descriptivos}
\noindent
Para esta parte se ejecutaron todos los cálculos descritos en la Fase 4 de la metodología realizando la implementación en RStudio, se obtuvo la tabla 3 con los resultados del análisis para las 6 regiones y para los 2 datasets (totales y magnitudes máximas). Se obtiene los valores de mínimo, máximo, media, mediana, moda, desviación estándar, varianza, asimetría, curtosis y total de sismos para cada región.

\begingroup
\footnotesize
\centering
\setlength{\LTcapwidth}{\textwidth}
\begin{longtable}{|>{\centering\arraybackslash}p{1.8cm}|>{\centering\arraybackslash}p{0.9cm}|>{\centering\arraybackslash}p{0.9cm}|>{\centering\arraybackslash}p{1cm}|>{\centering\arraybackslash}p{1.1cm}|>{\centering\arraybackslash}p{0.9cm}|>{\centering\arraybackslash}p{0.9cm}|>{\centering\arraybackslash}p{0.9cm}|>{\centering\arraybackslash}p{0.9cm}|>{\centering\arraybackslash}p{0.9cm}|>{\centering\arraybackslash}p{1cm}|}
\caption{Estadísticos descriptivos consolidados.} \label{tab:estadisticos_consolidados} \\
\hline
\rowcolor{gray!60}
\textbf{Región} & \textbf{Min} & \textbf{Max} & \textbf{Media} & \textbf{Med.} & \textbf{Moda} & \textbf{Desv.} & \textbf{Var.} & \textbf{Asim.} & \textbf{Curt.} & \textbf{Total} \\
\hline
\endfirsthead

\hline
\rowcolor{gray!60}
\textbf{Región} & \textbf{Min} & \textbf{Max} & \textbf{Media} & \textbf{Med.} & \textbf{Moda} & \textbf{Desv.} & \textbf{Var.} & \textbf{Asim.} & \textbf{Curt.} & \textbf{Total} \\
\hline
\endhead

\hline
\endfoot

\hline
\endlastfoot

\rowcolor{gray!20}
Oaxaca & 5.0 & 7.8 & 5.445 & 5.20 & 5.0 & 0.618 & 0.382 & 1.935 & 6.092 & 335 \\ \hline
Oax Max & 5.0 & 7.8 & 6.222 & 6.00 & 6.9 & 0.787 & 0.619 & 0.230 & 1.811 & 64 \\ \hline
\rowcolor{gray!20}
Guerrero & 5.0 & 7.8 & 5.573 & 5.30 & 5.0 & 0.695 & 0.483 & 1.390 & 3.854 & 256 \\ \hline
Gro Max & 5.0 & 7.8 & 6.289 & 6.50 & 6.6 & 0.794 & 0.631 & 0.028 & 1.773 & 70 \\ \hline
\rowcolor{gray!20}
Michoacán & 5.0 & 8.1 & 5.610 & 5.30 & 5.0 & 0.797 & 0.636 & 1.479 & 4.024 & 90 \\ \hline
Mich Max & 5.0 & 8.1 & 5.888 & 5.45 & 5.0 & 0.920 & 0.847 & 0.860 & 2.363 & 50 \\ \hline
\rowcolor{gray!20}
Chiapas & 5.0 & 8.2 & 5.395 & 5.20 & 5.0 & 0.559 & 0.313 & 2.114 & 7.373 & 629 \\ \hline
Chis Max & 5.1 & 8.2 & 6.442 & 6.50 & 5.6 & 0.739 & 0.547 & 0.005 & 2.221 & 73 \\ \hline
\rowcolor{gray!20}
Resto Nals. & 5.0 & 8.2 & 5.568 & 5.30 & 5.0 & 0.612 & 0.375 & 1.401 & 4.376 & 545 \\ \hline
R.N. Max & 5.3 & 8.2 & 6.517 & 6.50 & 6.5 & 0.579 & 0.335 & 0.033 & 3.176 & 87 \\ \hline
\rowcolor{gray!20}
Nacionales & 5.0 & 8.2 & 5.490 & 5.20 & 5.0 & 0.623 & 0.389 & 1.715 & 5.340 & 1855 \\ \hline
Nals. Max & 5.6 & 8.2 & 7.007 & 7.00 & 7.0 & 0.506 & 0.256 & -0.05 & 2.946 & 110 \\ \hline
\end{longtable}
\endgroup

De la tabla anterior se puede visualizar que el comportamiento de los sismos para una misma región iguales o mayores a 5° cambia significativamente cuando se analizan todos los sismos a cuando se hace un análisis de sismos de magnitud máxima anual. En todas las regiones se aprecia un aumento de la media cuando se manejan los datos máximos, así como de la mediana. Para la moda hay estados como Michoacán que no presentan diferencias entre su moda para sismos totales y su moda para magnitudes máximas. Para la varianza y desviación estándar también se aprecia un aumento en los valores cuando se trata de magnitudes máximas, salvo para las regiones de Resto Nacionales y Sismos Nacionales, donde se aprecia una menor varianza. Para la asimetría se encuentra que todas las regiones muestran asimetría positiva, mostrando que en todas las regiones se presentan eventos sísmicos de moderados a fuertes, destacándose Chiapas para los sismos totales con una cola muy marcada hacia la derecha. La asimetría para las magnitudes máximas presenta resultados más pequeños indicando menor sesgo positivo, salvo por sismos nacionales que presenta un sesgo negativo indicando mayor cantidad de valores bajos respecto de la media. La curtosis indica para los sismos totales que hay una mayor concentración de valores entorno a la media ósea que son leptocúrticas. Para los valores máximos los resultados indican que Oaxaca, Guerrero y Michoacán presentan distribuciones platicúrticas, mientras que Chiapas, Resto Nacional y Sismos Nacionales presentan distribuciones muy cercanas a una normal.

\section{Representación gráfica de los datos sísmicos}
\noindent
El obtener gráficos representativos de los datos sísmicos analizados para cada región, tanto para sismos totales como para magnitudes máximas, permite comprender y analizar de manera visual y clara el comportamiento de los sismos. A continuación, se presentan los diferentes histogramas y gráficos obtenidos.

\subsection{Histogramas de densidad de los sismos para sismos totales}