\chapter{Análisis de resultados \label{cap:AnalisisDeResultados}}

\noindent
El proceso de analizar y estudiar los datos sísmicos recopilados del SSN y luego aplicar la metodología descrita en el capítulo anterior nos lleva a la obtención de distintos resultados con los cuales se puede realizar la inferencia estadística de los sismos en las regiones estudiadas. A continuación, se presentan los principales resultados obtenidos después de aplicar los conocimientos descritos en el Capítulo 4 y se procede a realizar un análisis de los mismos para obtener resultados confiables basados en las tendencias de cada región.

\section{Herramientas utilizadas}

El archivo recopilado desde el sitio del SSN es un archivo en formato csv bruto el cual contiene la información estadística de todos los sismos del país desde el 01 de enero de 1900 hasta una fecha determinada. En este caso se esta trabajando con un archivo que contiene datos sísmicos hasta el 23 de julio de 2025.

En la metodología, la Fase 2 se realiza utilizando un código desarrollado en Python para hacer el filtrado del archivo de sismos generales y para obtener los archivos individuales de cada región en formato xlsx. El código se ejecuta en Google Colab a través de servidores en la nube y el resultado son 6 archivos de Excel con la información de datos sísmicos depurada de sismos iguales o mayores a 5°.

El análisis de los archivos individuales de sismos y todos los cálculos relacionados al análisis estadístico se realizan en la herramienta R (version 4.4.2) y RStudio (2024.09.1 Build 394).

\section{Resultados de estadísticos descriptivos}

Para esta parte se ejecutaron todos los cálculos descritos en la Fase 4 de la metodología realizando la implementación en RStudio, se obtuvo la tabla 2 con los resultados del análisis para las 6 regiones y para los 2 datasets (totales y magnitudes máximas).

{  %% Se agrupa el código para que afecte solo a esta tabla
\footnotesize
\centering
\begin{tabularx}{15.9cm}{|>{\centering\arraybackslash}p{2cm}|* {9}{>{\centering\arraybackslash}X|} >{\centering\arraybackslash}p{1cm}|}
\caption{Estadísticos descriptivos consolidados.} \label{tab:estadisticos_consolidados} \\
\hline
\rowcolor{gray!60}
\textbf{Región} & \textbf{Min} & \textbf{Max} & \textbf{Media} & \textbf{Mediana} & \textbf{Moda} & \textbf{Desv.} & \textbf{Var.} & \textbf{Asim.} & \textbf{Curt.} & \textbf{Totales} \\
\hline
\rowcolor{gray!20}
Oaxaca & 5.0 & 7.8 & 5.445 & 5.20 & 5.0 & 0.618 & 0.382 & 1.935 & 6.092 & 335 \\ \hline
Oax Max & 5.0 & 7.8 & 6.222 & 6.00 & 6.9 & 0.787 & 0.619 & 0.230 & 1.811 & 64 \\ \hline
\rowcolor{gray!20}
Guerrero & 5.0 & 7.8 & 5.573 & 5.30 & 5.0 & 0.695 & 0.483 & 1.390 & 3.854 & 256 \\ \hline
Gro Max & 5.0 & 7.8 & 6.289 & 6.50 & 6.6 & 0.794 & 0.631 & 0.028 & 1.773 & 70 \\ \hline
\rowcolor{gray!20}
Michoacán & 5.0 & 8.1 & 5.610 & 5.30 & 5.0 & 0.797 & 0.636 & 1.479 & 4.024 & 90 \\ \hline
Mich Max & 5.0 & 8.1 & 5.888 & 5.45 & 5.0 & 0.920 & 0.847 & 0.860 & 2.363 & 50 \\ \hline
\rowcolor{gray!20}
Chiapas & 5.0 & 8.2 & 5.395 & 5.20 & 5.0 & 0.559 & 0.313 & 2.114 & 7.373 & 629 \\ \hline
Chis Max & 5.1 & 8.2 & 6.442 & 6.50 & 5.6 & 0.739 & 0.547 & 0.005 & 2.221 & 73 \\ \hline
\rowcolor{gray!20}
Resto Nals. & 5.0 & 8.2 & 5.568 & 5.30 & 5.0 & 0.612 & 0.375 & 1.401 & 4.376 & 545 \\ \hline
R.N. Max & 5.3 & 8.2 & 6.517 & 6.50 & 6.5 & 0.579 & 0.335 & 0.033 & 3.176 & 87 \\ \hline
\rowcolor{gray!20}
Nacionales & 5.0 & 8.2 & 5.490 & 5.20 & 5.0 & 0.623 & 0.389 & 1.715 & 5.340 & 1855 \\ \hline
Nals. Max & 5.6 & 8.2 & 7.007 & 7.00 & 7.0 & 0.506 & 0.256 & -0.050 & 2.946 & 110 \\ \hline
\end{tabularx}
}  %% Se cierra el grupo

De la tabla anterior se puede visualizar que el comportamiento de los sismos para una misma región iguales o mayores a 5° cambia significativamente cuando se analizan todos los sismos a cuando se hace un análisis de sismos de magnitud máxima anual. En todas las regiones se aprecia un aumento de la media cuando se manejan los datos máximos, así como de la mediana. Para la moda hay estados como Michoacán que no presentan diferencias entre su moda para sismos totales y su moda para magnitudes máximas. Para la varianza y desviación estándar también se aprecia un aumento en los valores cuando se trata de magnitudes máximas, salvo para las regiones de Resto Nacionales y Sismos Nacionales, donde se aprecia una menor varianza. Para la asimetría se encuentra que todas las regiones muestran asimetría positiva, mostrando que en todas las regiones se presentan eventos sísmicos de moderados a fuertes, destacándose Chiapas para los sismos totales con una cola muy marcada hacia la derecha. La asimetría para las magnitudes máximas presenta resultados más pequeños indicando menor sesgo positivo, salvo por sismos nacionales que presenta un sesgo negativo indicando mayor cantidad de valores bajos respecto de la media. La curtosis indica para los sismos totales que hay una mayor concentración de valores entorno a la media ósea que son leptocúrticas. Para los valores máximos los resultados indican que Oaxaca, Guerrero y Michoacán presentan distribuciones platicúrticas, mientras que Chiapas, Resto Nacional y Sismos Nacionales presentan distribuciones muy cercanas a una normal.

\section{Representación gráfica de los datos sísmicos}

El obtener gráficos representativos de los datos sísmicos analizados para cada región, tanto para sismos totales como para magnitudes máximas, permite comprender y analizar de manera visual y clara el comportamiento de los sismos. A continuación, se presentan los diferentes histogramas y gráficos obtenidos.

\subsection{Histogramas de densidad de los sismos para sismos totales}

\begin{figure}[H]
\centering
\includegraphics[width=\textwidth]{histogramas_densidad_totales.png}
\caption{Histogramas de densidad de sismos totales para las 6 regiones.}
\label{fig:histogramas_totales}
\end{figure}

Observando los histogramas, se puede realizar el siguiente análisis para los datos de sismos totales.

\begin{itemize}
\item Oaxaca: presenta una densidad de sismos marcada entre 5.0° y 5.9°, con un pico aislado en 6.8°, indicando predominancia de eventos de magnitud baja.

\item Guerrero: presenta una alta densidad de eventos entre 5.0° y 5.7°, con picos menores en 5.9° y 6.6°, lo cual indica un comportamiento sísmico con eventos de magnitud moderada, pero con presencia intermitente de eventos considerables.

\item Michoacán: presenta la mayor densidad de sismos en la escala de 5.0° a 5.5°, pero tiene varios picos significativos en 5.8°, 6.8° y 7.5°, lo cual indica una mayor dispersión en la distribución de los sismos, con posible presencia de eventos de magnitudes muy fuertes o extremas.

\item Chiapas: presenta la mayor densidad de sismos en la escala de 5.0° a 5.6°, con una caída uniforme conforme aumenta la magnitud. Esto indica un comportamiento sísmico más predecible y donde se presentarán mayoritariamente eventos de magnitud moderada.

\item Resto Nacionales: en el resto de los estados del país (sin incluir los 4 estados anteriores), se presenta una distribución más uniforme de la densidad de sismos respecto a la de los estados, conteniendo la mayor densidad de los sismos entre 5.0° y 5.6°, con un pico en 6.4°. Esto sugiere que hay pocos eventos de magnitud considerable o extrema y que normalmente se presentan sismos de magnitud moderada a baja.

\item Sismos Nacionales: el análisis de la densidad de sismos de los 32 estados del país presenta una curva de actividad sísmica más uniforme que la de los estados individuales, con la mayoría de los sismos en la escala de 5.0° a 5.6°, indicando que la mayoría de los eventos sísmicos ocurridos en México son de magnitud baja.
\end{itemize}

\subsection{Histogramas de densidad de los sismos para sismos totales}

\begin{figure}[H]
\centering
\includegraphics[width=\textwidth]{histogramas_densidad_maximas.png}
\caption{Histogramas de densidad de magnitudes máximas para las 6 regiones.}
\label{fig:histogramas_maximas}
\end{figure}

Observando los histogramas, se puede realizar el siguiente análisis para los datos de magnitudes máximas.

\begin{itemize}
\item Oaxaca: en el análisis de magnitudes máximas se observan picos de densidad de sismos en 5.0°, 5.5° y 6.8°, lo cual indica que en el análisis de magnitudes máximas se observa que la gran mayoría de sismos fuertes en Oaxaca son menores a 7.0°.

\item Guerrero: en el análisis de magnitudes máximas se observa que los picos de densidad de sismos están en la escala de 6.4° y 6.5°, indicando que en el análisis de magnitudes máximas se observa que la gran mayoría de sismos fuertes en Guerrero también son menores a 7.0°.

\item Michoacán: en el análisis de magnitudes máximas se observa un pico de densidad de sismos en la escala de 5.0°, indicando que en el análisis de magnitudes máximas se observa que la gran mayoría de sismos fuertes en Michoacán suelen ser menores a 6.0°, aunque también se presenta una dispersión de sismos que tiene un pico en 7.5°, lo cual identifica al estado de Michoacán como una zona de alta variabilidad sísmica.

\item Chiapas: en el análisis de magnitudes máximas se observan picos de densidad de sismos en 5.5° y 6.5°, indicando que en el análisis de magnitudes máximas se observa que la gran mayoría de sismos fuertes en Chiapas están en el rango de 5.5° a 7.2°, con lo cual se observa una intensa actividad de sismos fuertes para este estado.

\item Resto Nacionales: en el análisis de magnitudes máximas para el resto de los estados del país se observan picos de densidad de sismos en 6.4°, 6.6° y 7.1°. Indicando que en el resto del país la mayoría de los sismos fuertes se encuentra en un rango de 6.1° a 7.0°.

\item Sismos Nacionales: en el análisis de magnitudes máximas para México se observan picos de densidad de sismos en 6.9° y 7.0°. Indicando que en México la mayoría de los sismos fuertes se encuentra en un rango de 6.4° a 7.5°.
\end{itemize}

\subsection{Histogramas de frecuencias relativas por mes para sismos totales}

\begin{figure}[H]
\centering
\includegraphics[width=\textwidth]{frecuencias_mes_totales.png}
\caption{Histogramas de Frecuencias Relativas por Mes para sismos totales para las 6 regiones.}
\label{fig:frecuencias_mes_totales}
\end{figure}

Analizando los histogramas de frecuencias relativas por mes para sismos totales, se puede realizar el análisis siguiente del comportamiento temporal de la actividad sísmica en las distintas regiones de análisis.

\begin{itemize}
\item Oaxaca: presenta que a lo largo del tiempo el mes más activo es Septiembre, mientras que el mes con menos registros sísmicos para este estado es Octubre.

\item Guerrero: presenta que el mes más activo sísmicamente es Mayo, mientras que el mes con menor actividad es Febrero.

\item Michoacán: presenta que el mes con más sismos es Enero, mientras que el mes con menor actividad es Noviembre.

\item Chiapas: presenta que el mes más activo es Septiembre (caso similar a Oaxaca), mientras que el resto de los meses del año presentan una frecuencia muy similar, siendo Julio el mes con menor actividad.

\item Resto Nacionales: presenta que Octubre es el mes con mayor actividad sísmica para el resto del país, mientras que Junio es el mes con menor actividad sísmica.

\item Sismos Nacionales: presenta que Septiembre es el mes con mayor actividad en el país, mientras que Julio es el mes con menor actividad sísmica en México.
\end{itemize}

\subsection{Histogramas de frecuencias relativas por mes para magnitudes máximas}

\begin{figure}[H]
\centering
\includegraphics[width=\textwidth]{frecuencias_mes_maximas.png}
\caption{Histogramas de Frecuencias Relativas por Mes para magnitudes máximas para las 6 regiones.}
\label{fig:frecuencias_mes_maximas}
\end{figure}

Analizando los histogramas de frecuencias relativas por mes para magnitudes máximas, se puede realizar el análisis siguiente del comportamiento temporal de la actividad sísmica en las distintas regiones de análisis.

\begin{itemize}
\item Oaxaca: presenta que a lo largo del tiempo el mes con más actividad sísmica fuerte es Junio, mientras que el mes con menos registros sísmicos fuertes es Octubre.

\item Guerrero: presenta que el mes con más sismos fuertes es Julio, mientras que el mes con menor actividad sísmica fuerte es Agosto.

\item Michoacán: presenta que el mes con más sismos fuertes es Enero, mientras que Febrero no presenta valores de registros sísmicos, indicando que no se han presentado sismos fuertes en este mes.

\item Chiapas: presenta que el mes más activo para sismos fuertes es Diciembre, mientras que el mes con menor actividad sísmica fuerte es Abril.

\item Resto Nacionales: presenta que Mayo es el mes con mayor actividad sísmica fuerte para el resto del país, mientras que Marzo es el mes con menor actividad sísmica fuerte.

\item Sismos Nacionales: presenta que Septiembre es el mes con mayor actividad sísmica fuerte en el país, mientras que Noviembre es el mes con menor actividad sísmica fuerte en México.
\end{itemize}

\subsection{Histogramas de sismos por magnitud}

\begin{figure}[H]
\centering
\includegraphics[width=\textwidth]{histogramas_magnitud.png}
\caption{Histogramas de sismos por magnitud para las 6 regiones.}
\label{fig:histogramas_magnitud}
\end{figure}

Analizando los histogramas de sismos por magnitud, se puede realizar el análisis siguiente.

\begin{itemize}
\item Oaxaca: presenta una elevada concentración de sismos de magnitud 5.0° y alcanza un máximo registrado de 7.8°.

\item Guerrero: presenta una elevada concentración de sismos de magnitud 5.0° y también alcanza un máximo registrado de 7.8°.

\item Michoacán: presenta una moderada concentración de sismos de magnitud 5.0° y alcanza un máximo registrado de 8.1°.

\item Chiapas: presenta una muy elevada concentración de sismos de magnitud 5.0° y alcanza un máximo registrado de 8.2°.

\item Resto Nacionales: presenta una elevada concentración de sismos de magnitud 5.0°, 5.1° y 5.2° y alcanza un máximo registrado de 8.2°.

\item Sismos Nacionales: presenta una elevada concentración de sismos de magnitud 5.0° y 5.1° y alcanza un máximo registrado de 8.2°.
\end{itemize}

\subsection{Gráfico de magnitud máxima, promedio y mínima en el tiempo}

\begin{figure}[H]
\centering
\includegraphics[width=\textwidth]{magnitudes_tiempo.png}
\caption{Gráficos de magnitud sísmica máxima, promedio y mínima para las 6 regiones.}
\label{fig:magnitudes_tiempo}
\end{figure}

Analizando los gráficos para las 6 regiones se puede observar que antes de los años 70s del siglo pasado, las líneas de magnitud máxima, promedio y mínima no presentaban prácticamente variaciones para los 4 estados, y presentan muy poca variación para Resto Nacional y Sismos Nacionales en este periodo. Lo cual indica que hubo una progresiva mejoría en las técnicas de medición y registro de eventos sísmicos. También podemos observar sismos de interés en los gráficos de Michoacán y Chiapas, siendo el sismo de septiembre 19, 1985 el que aparece como un pico en el gráfico de Michoacán, mientras que para el de Chiapas se aprecia el sismo del 07 de septiembre, 2017 el que aparece claramente en su gráfico y que además resulta ser el sismo más fuerte registrado en toda la base de datos sísmicos de México.
\section{Intervalos de confianza y tamaño mínimo de la muestra}

La estimación de Intervalos de Confianza (IC) constituye una herramienta fundamental para el análisis estadístico al estimar la incertidumbre asociada a parámetros como la media, la desviación estándar y la proporción. Los rangos que se obtienen permiten visualizar en qué valores es probable que se encuentren los parámetros verdaderos con un nivel de confianza (para este trabajo, se especifica en 95\%).

\subsection{Intervalo de confianza para la media}

Se presenta la tabla 3 con los IC para la media para sismos totales para las 6 regiones de estudio.

%%\small
\footnotesize
\begin{table}[h!]
\centering
\caption{IC para la media para sismos totales para las 6 regiones.} 
\label{tab:ic_media_totales}
\begin{tabular}{|c|c|c|c|}
\hline
\rowcolor{gray!60}
\textbf{Estado} & \textbf{Media Totales} & \textbf{LI 95\% Totales} & \textbf{LS 95\% Totales} \\ \hline
\rowcolor{gray!20}
Oaxaca & 5.445075 & 5.378654 & 5.511495 \\ \hline
Guerrero & 5.573437 & 5.487865 & 5.659010 \\ \hline
\rowcolor{gray!20}
Michoacán & 5.610000 & 5.443002 & 5.776998 \\ \hline
Chiapas & 5.395072 & 5.351273 & 5.438870 \\ \hline
\rowcolor{gray!20}
Resto Nacionales & 5.567706 & 5.516171 & 5.619242 \\ \hline
Nacionales & 5.489865 & 5.461474 & 5.518256 \\ \hline
\end{tabular}
\end{table}

Ahora se presenta la figura 7 con los gráficos de los IC para la media para sismos totales para las 6 regiones de estudio.

\begin{figure}[H]
\centering
\includegraphics[width=0.8\textwidth]{ic_media_totales.png}
\caption{IC para la media para sismos totales para las 6 regiones.}
\label{fig:ic_media_totales}
\end{figure}

A continuación, se presenta la tabla 4 con los IC para la media para magnitudes máximas para las 6 regiones de estudio.

%%\small
\footnotesize
\begin{table}[h!]
\centering
\caption{IC para la media para magnitudes máximas para las 6 regiones.} 
\label{tab:ic_media_maximas}
\begin{tabular}{|c|c|c|c|}
\hline
\rowcolor{gray!60}
\textbf{Estado} & \textbf{Media Máximos} & \textbf{LI 95\% Máximos} & \textbf{LS 95\% Máximos} \\ \hline
\rowcolor{gray!20}
Oaxaca & 6.221875 & 6.025316 & 6.418434 \\ \hline
Guerrero & 6.288571 & 6.099182 & 6.477961 \\ \hline
\rowcolor{gray!20}
Michoacán & 5.888000 & 5.626478 & 6.149522 \\ \hline
Chiapas & 6.442466 & 6.269962 & 6.614970 \\ \hline
\rowcolor{gray!20}
Resto Nacionales & 6.517241 & 6.393854 & 6.640629 \\ \hline
Nacionales & 7.007273 & 6.911676 & 7.102870 \\ \hline
\end{tabular}
\end{table}

Ahora se presenta la figura 8 con los gráficos de los IC para la media para magnitudes máximas para las 6 regiones de estudio.

\begin{figure}[H]
\centering
\includegraphics[width=0.8\textwidth]{ic_media_maximas.png}
\caption{IC para la media para magnitudes máximas para las 6 regiones.}
\label{fig:ic_media_maximas}
\end{figure}

\subsection{Intervalo de confianza para la varianza}

Se presenta la tabla 5 con los IC para la varianza para sismos totales para las 6 regiones de estudio.

%%\small
\footnotesize
\begin{table}[h!]
\centering
\caption{Tabla 5. IC para la varianza para sismos totales para las 6 regiones.} 
\label{tab:ic_varianza_totales}
\begin{tabular}{|c|c|c|c|}
\hline
\rowcolor{gray!60}
\textbf{Estado} & \textbf{Varianza Totales} & \textbf{LI 95\% Var\_Tot} & \textbf{LS 95\% Var\_Tot} \\ \hline
\rowcolor{gray!20}
Oaxaca & 0.3819442 & 0.3300430 & 0.4471985 \\ \hline
Guerrero & 0.4833701 & 0.4093295 & 0.5796038 \\ \hline
\rowcolor{gray!20}
Michoacán & 0.6357416 & 0.4836434 & 0.8732530 \\ \hline
Chiapas & 0.3128897 & 0.2809734 & 0.3506005 \\ \hline
\rowcolor{gray!20}
Resto Nacionales & 0.3751317 & 0.3342576 & 0.4240313 \\ \hline
Nacionales & 0.3887214 & 0.3648633 & 0.4150106 \\ \hline
\end{tabular}
\end{table}

Ahora se presenta la figura 9 con los gráficos de los IC para la varianza para sismos totales para las 6 regiones de estudio.

\begin{figure}[H]
\centering
\includegraphics[width=0.8\textwidth]{ic_varianza_totales.png}
\caption{IC para la varianza para sismos totales para las 6 regiones.}
\label{fig:ic_varianza_totales}
\end{figure}

Se presenta la tabla 6 con los IC para la varianza para magnitudes máximas para las 6 regiones de estudio.

%%\small
\footnotesize
\begin{table}[h!]
\centering
\caption{IC para la varianza para magnitudes máximas para las 6 regiones.} 
\label{tab:ic_varianza_maximas}
\begin{tabular}{|c|c|c|c|}
\hline
\rowcolor{gray!60}
\textbf{Estado} & \textbf{Varianza Máximos} & \textbf{LI 95\% Var\_Max} & \textbf{LS 95\% Var\_Max} \\ \hline
\rowcolor{gray!20}
Oaxaca & 0.6191964 & 0.4492636 & 0.9082451 \\ \hline
Guerrero & 0.6308820 & 0.4638024 & 0.9083280 \\ \hline
\rowcolor{gray!20}
Michoacán & 0.8467918 & 0.5908769 & 1.3149393 \\ \hline
Chiapas & 0.5466438 & 0.4042848 & 0.7804875 \\ \hline
\rowcolor{gray!20}
Resto Nacionales & 0.3351644 & 0.2538596 & 0.4631230 \\ \hline
Nacionales & 0.2559099 & 0.1995522 & 0.3401743 \\ \hline
\end{tabular}
\end{table}

Ahora se presenta la figura 10 con los gráficos de los IC para la varianza para sismos totales para las 6 regiones de estudio.

\begin{figure}[H]
\centering
\includegraphics[width=0.8\textwidth]{ic_varianza_maximas.png}
\caption{IC para la varianza para magnitudes máximas para las 6 regiones.}
\label{fig:ic_varianza_maximas}
\end{figure}

\subsection{Intervalo de confianza para la proporción de sismos mayores al umbral crítico}

Para la estimación de los IC de proporción, se maneja un umbral crítico de 6.5° y se analizará que proporción de sismos superan ese umbral. Se presenta la tabla 7 con los IC para proporción de sismos mayores al umbral crítico para sismos totales para las 6 regiones de estudio.

%%\small
\footnotesize
\centering
\begin{tabularx}{15.9cm}{|>{\centering\arraybackslash}p{2.2cm}|* {6}{>{\centering\arraybackslash}X|}}
\caption{IC para la proporción de sismos mayores al umbral crítico para sismos totales para las 6 regiones.} \label{tab:ic_proporcion_totales} \\
\hline
\rowcolor{gray!60}
\textbf{Estado} & \textbf{Proporción Totales} & \textbf{Porcentaje Totales} & \textbf{LI 95\% Prop\_Tot} & \textbf{LS 95\% Prop\_Tot} & \textbf{LI 95\% Porc\_Tot} & \textbf{LS 95\% Porc\_Tot} \\ \hline
\rowcolor{gray!20}
Oaxaca & 0.0896 & 8.96 & 0.0590 & 0.1202 & 5.90 & 12.02 \\ \hline
Guerrero & 0.1445 & 14.45 & 0.1014 & 0.1876 & 10.14 & 18.76 \\ \hline
\rowcolor{gray!20}
Michoacán & 0.1667 & 16.67 & 0.0897 & 0.2437 & 8.97 & 24.37 \\ \hline
Chiapas & 0.0700 & 7.00 & 0.0501 & 0.0899 & 5.01 & 8.99 \\ \hline
\rowcolor{gray!20}
Resto Nals. & 0.1009 & 10.09 & 0.0756 & 0.1262 & 7.56 & 12.62 \\ \hline
Nacionales & 0.0976 & 9.76 & 0.0841 & 0.1111 & 8.41 & 11.11 \\ \hline
\end{tabularx}

Ahora se presenta la figura 11 con los gráficos de los IC para la proporción de sismos mayores al umbral crítico (6.5°) para sismos totales para las 6 regiones de estudio.

\begin{figure}[H]
\centering
\includegraphics[width=0.8\textwidth]{ic_proporcion_totales.png}
\caption{IC para la proporción de sismos mayores al umbral crítico para sismos totales para las 6 regiones.}
\label{fig:ic_proporcion_totales}
\end{figure}

Se presenta la tabla 8 con los IC para proporción de sismos mayores al umbral crítico para magnitudes máximas para las 6 regiones de estudio.

%%\small
\footnotesize
\centering
\begin{tabularx}{15.9cm}{|>{\centering\arraybackslash}p{2.2cm}|* {6}{>{\centering\arraybackslash}X|}}
\caption{IC para la proporción de sismos mayores al umbral crítico para magnitudes máximas para las 6 regiones.} \label{tab:ic_proporcion_maximas} \\
\hline
\rowcolor{gray!60}
\textbf{Estado} & \textbf{Proporción Máximos} & \textbf{Porcentaje Máximos} & \textbf{LI 95\% Prop\_Max} & \textbf{LS 95\% Prop\_Max} & \textbf{LI 95\% Porc\_Max} & \textbf{LS 95\% Porc\_Max} \\ \hline
\rowcolor{gray!20}
Oaxaca & 0.3750 & 37.50 & 0.2564 & 0.4936 & 25.64 & 49.36 \\ \hline
Guerrero & 0.4429 & 44.29 & 0.3265 & 0.5593 & 32.65 & 55.93 \\ \hline
\rowcolor{gray!20}
Michoacán & 0.2800 & 28.00 & 0.1555 & 0.4045 & 15.55 & 40.45 \\ \hline
Chiapas & 0.4932 & 49.32 & 0.3785 & 0.6079 & 37.85 & 60.79 \\ \hline
\rowcolor{gray!20}
Resto Nals. & 0.4828 & 48.28 & 0.3778 & 0.5878 & 37.78 & 58.78 \\ \hline
Nacionales & 0.8182 & 81.82 & 0.7461 & 0.8903 & 74.61 & 89.03 \\ \hline
\end{tabularx}

Ahora se presenta la figura 12 con los gráficos de los IC para la proporción de sismos mayores al umbral crítico para magnitudes máximas para las 6 regiones de estudio.

\begin{figure}[H]
\centering
\includegraphics[width=0.8\textwidth]{ic_proporcion_maximas.png}
\caption{IC para la proporción de sismos mayores al umbral crítico para magnitudes máximas para las 6 regiones.}
\label{fig:ic_proporcion_maximas}
\end{figure}
\subsection{Resultados de las pruebas de proporciones mayores a 6.5°}

Una vez obtenidos los IC para las proporciones de sismos mayores al umbral crítico (6.5°) tanto para sismos totales como para magnitudes máximas, se presentan los resultados del análisis de estos.

En la tabla 9 se muestran los resultados de las comparaciones entre regiones de proporción de sismos mayores al umbral crítico para todos los sismos.

%%\small
\footnotesize
\centering
\begin{tabularx}{15.9cm}{|c|c|* {5}{>{\centering\arraybackslash}X|}c|}
\caption{Comparación entre todas las regiones de la proporción de sismos mayor al umbral crítico para sismos totales.} \label{tab:comparacion_proporciones_totales} \\
\hline
\rowcolor{gray!60}
\textbf{Estado 1} & \textbf{Estado 2} & \textbf{Prop\_1} & \textbf{Prop\_2} & \textbf{Diferencia} & \textbf{Z\_est.} & \textbf{P\_valor} & \textbf{Significativo} \\ \hline
\rowcolor{gray!20}
Oaxaca & Guerrero & 0.0896 & 0.1445 & -0.0549 & -2.037 & 0.0417 & Sí \\ \hline
Oaxaca & Michoacán & 0.0896 & 0.1667 & -0.0771 & -1.824 & 0.0682 & No \\ \hline
\rowcolor{gray!20}
Oaxaca & Chiapas & 0.0896 & 0.0700 & 0.0196 & 1.052 & 0.2927 & No \\ \hline
Oaxaca & Resto Nals. & 0.0896 & 0.1009 & -0.0113 & -0.558 & 0.5768 & No \\ \hline
\rowcolor{gray!20}
Oaxaca & Nacionales & 0.0896 & 0.0976 & -0.0080 & -0.469 & 0.6391 & No \\ \hline
Guerrero & Michoacán & 0.1445 & 0.1667 & -0.0222 & -0.493 & 0.6219 & No \\ \hline
\rowcolor{gray!20}
Guerrero & Chiapas & 0.1445 & 0.0700 & 0.0745 & 3.077 & 0.0021 & Sí \\ \hline
Guerrero & Resto Nals. & 0.1445 & 0.1009 & 0.0436 & 1.711 & 0.0871 & No \\ \hline
\rowcolor{gray!20}
Guerrero & Nacionales & 0.1445 & 0.0976 & 0.0469 & 2.036 & 0.0417 & Sí \\ \hline
Michoacán & Chiapas & 0.1667 & 0.0700 & 0.0967 & 2.383 & 0.0172 & Sí \\ \hline
\rowcolor{gray!20}
Michoacán & Resto Nals. & 0.1667 & 0.1009 & 0.0658 & 1.591 & 0.1116 & No \\ \hline
Michoacán & Nacionales & 0.1667 & 0.0976 & 0.0691 & 1.732 & 0.0832 & No \\ \hline
\rowcolor{gray!20}
Chiapas & Resto Nals. & 0.0700 & 0.1009 & -0.0309 & -1.881 & 0.0600 & No \\ \hline
Chiapas & Nacionales & 0.0700 & 0.0976 & -0.0276 & -2.246 & 0.0247 & Sí \\ \hline
\rowcolor{gray!20}
Resto Nals. & Nacionales & 0.1009 & 0.0976 & 0.0033 & 0.226 & 0.8215 & No \\ \hline
\end{tabularx}

En la tabla 10 se muestran los resultados de las comparaciones entre regiones de proporción de sismos mayores al umbral crítico para magnitudes máximas.

%%\small
\footnotesize
\centering
\begin{tabularx}{15.9cm}{|c|c|* {5}{>{\centering\arraybackslash}X|}c|}
\caption{Comparación entre todas las regiones de la proporción de sismos mayor al umbral crítico para magnitudes máximas.} \label{tab:comparacion_proporciones_maximas} \\
\hline
\rowcolor{gray!60}
\textbf{Estado 1} & \textbf{Estado 2} & \textbf{Prop\_1} & \textbf{Prop\_2} & \textbf{Diferencia} & \textbf{Z\_est.} & \textbf{P\_valor} & \textbf{Significativo} \\ \hline
\rowcolor{gray!20}
Oaxaca & Guerrero & 0.3750 & 0.4429 & -0.0679 & -0.801 & 0.4232 & No \\ \hline
Oaxaca & Michoacán & 0.3750 & 0.2800 & 0.0950 & 1.083 & 0.2788 & No \\ \hline
\rowcolor{gray!20}
Oaxaca & Chiapas & 0.3750 & 0.4932 & -0.1182 & -1.404 & 0.1603 & No \\ \hline
Oaxaca & Resto Nals. & 0.3750 & 0.4828 & -0.1078 & -1.334 & 0.1823 & No \\ \hline
\rowcolor{gray!20}
Oaxaca & Nacionales & 0.3750 & 0.8182 & -0.4432 & -6.259 & 0.0000 & Sí \\ \hline
Guerrero & Michoacán & 0.4429 & 0.2800 & 0.1629 & 1.874 & 0.0609 & No \\ \hline
\rowcolor{gray!20}
Guerrero & Chiapas & 0.4429 & 0.4932 & -0.0503 & -0.603 & 0.5462 & No \\ \hline
Guerrero & Resto Nals. & 0.4429 & 0.4828 & -0.0399 & -0.499 & 0.6178 & No \\ \hline
\rowcolor{gray!20}
Guerrero & Nacionales & 0.4429 & 0.8182 & -0.3753 & -5.374 & 0.0000 & Sí \\ \hline
Michoacán & Chiapas & 0.2800 & 0.4932 & -0.2132 & -2.469 & 0.0135 & Sí \\ \hline
\rowcolor{gray!20}
Michoacán & Resto Nals. & 0.2800 & 0.4828 & -0.2028 & -2.441 & 0.0146 & Sí \\ \hline
Michoacán & Nacionales & 0.2800 & 0.8182 & -0.5382 & -7.335 & 0.0000 & Sí \\ \hline
\rowcolor{gray!20}
Chiapas & Resto Nals. & 0.4932 & 0.4828 & 0.0104 & 0.131 & 0.8957 & No \\ \hline
Chiapas & Nacionales & 0.4932 & 0.8182 & -0.3250 & -4.703 & 0.0000 & Sí \\ \hline
\rowcolor{gray!20}
Resto Nals. & Nacionales & 0.4828 & 0.8182 & -0.3354 & -5.162 & 0.0000 & Sí \\ \hline
\end{tabularx}

Ahora la figura 12 presenta dos listas decrecientes con el ranking de estados con mayor proporción de sismos mayores al umbral, tanto para sismos totales y magnitudes máximas:

\begin{figure}[H]
\centering
\begin{verbatim}
=== RANKING POR PROPORCIÓN DE SISMOS TOTALES MAYORES AL UMBRAL ===
1. Michoacán: 0.1667 (16.67%) – 15 de 90 sismos
2. Guerrero: 0.1445 (14.45%) – 37 de 256 sismos
3. Resto Nacionales: 0.1009 (10.09%) – 55 de 545 sismos
4. Nacional: 0.0976 (9.76%) – 181 de 1855 sismos
5. Oaxaca: 0.0896 (8.96%) – 30 de 335 sismos
6. Chiapas: 0.0700 (7.00%) – 44 de 629 sismos

=== RANKING POR PROPORCIÓN DE MAGNITUDES MÁXIMAS MAYORES AL UMBRAL ===
1. Nacional: 0.8182 (81.82%) – 90 de 110 registros
2. Chiapas: 0.4932 (49.32%) – 36 de 73 registros
3. Resto Nacionales: 0.4828 (48.28%) – 42 de 87 registros
4. Guerrero: 0.4429 (44.29%) – 31 de 70 registros
5. Oaxaca: 0.3750 (37.50%) – 24 de 64 registros
6. Michoacán: 0.2800 (28.00%) – 14 de 50 registros
\end{verbatim}
\caption{Ranking de regiones con mayor cantidad de sismos mayores al umbral crítico para sismos totales y magnitudes máximas.}
\label{fig:ranking_proporciones}
\end{figure}

\section{Estimación del tamaño mínimo de la muestra para la media}

El tamaño mínimo de la muestra indica el número menor de observaciones necesarias para estimar un parámetro poblacional (en este caso, la media) con un nivel de precisión específico y un grado de confianza determinado. Para esta prueba, se utilizan los siguientes parámetros:

\begin{itemize}
\item Nivel de confianza del 95\%
\item Un error en intervalos desde 1\% hasta 30\%
\end{itemize}

La tabla 11 muestra el tamaño mínimo de la muestra para las 6 regiones para sismos totales.

%%\small
\footnotesize
\centering
\begin{tabularx}{15.9cm}{|c|* {6}{>{\centering\arraybackslash}X|}}
\caption{Tamaño mínimo de la muestra para las 6 regiones para sismos totales.} \label{tab:tamano_muestra_totales} \\
\hline
\rowcolor{gray!60}
\textbf{$\epsilon$} & \textbf{Oaxaca (n=335)} & \textbf{Guerrero (n=256)} & \textbf{Michoacán (n=90)} & \textbf{Chiapas (n=629)} & \textbf{Resto Nacionales (n=545)} & \textbf{Sismos Nacionales (n=1855)} \\ \hline
\rowcolor{gray!20}
\textbf{0.01} & 14780 & 18746 & 25100 & 12066 & 14475 & 14953 \\ \hline
\textbf{0.02} & 3695 & 4687 & 6275 & 3017 & 3619 & 3739 \\ \hline
\rowcolor{gray!20}
\textbf{0.03} & 1643 & 2083 & 2789 & 1341 & 1609 & 1662 \\ \hline
\textbf{0.04} & 924 & 1172 & 1569 & 755 & 905 & 935 \\ \hline
\rowcolor{gray!20}
\textbf{0.05} & 592 & 750 & 1004 & 483 & 579 & 599 \\ \hline
\textbf{0.06} & 411 & 521 & 698 & 336 & 403 & 416 \\ \hline
\rowcolor{gray!20}
\textbf{0.07} & 302 & 383 & 513 & 247 & 296 & 306 \\ \hline
\textbf{0.08} & 231 & 293 & 393 & 189 & 227 & 234 \\ \hline
\rowcolor{gray!20}
\textbf{0.09} & 183 & 232 & 310 & 149 & 179 & 185 \\ \hline
\textbf{0.1} & 148 & 188 & 251 & 121 & 145 & 150 \\ \hline
\rowcolor{gray!20}
\textbf{0.11} & 123 & 155 & 208 & 100 & 120 & 124 \\ \hline
\textbf{0.12} & 103 & 131 & 175 & 84 & 101 & 104 \\ \hline
\rowcolor{gray!20}
\textbf{0.13} & 88 & 111 & 149 & 72 & 86 & 89 \\ \hline
\textbf{0.14} & 76 & 96 & 129 & 62 & 74 & 77 \\ \hline
\rowcolor{gray!20}
\textbf{0.15} & 66 & 84 & 112 & 54 & 65 & 67 \\ \hline
\textbf{0.16} & 58 & 74 & 99 & 48 & 57 & 59 \\ \hline
\rowcolor{gray!20}
\textbf{0.17} & 52 & 65 & 87 & 42 & 51 & 52 \\ \hline
\textbf{0.18} & 46 & 58 & 78 & 38 & 45 & 47 \\ \hline
\rowcolor{gray!20}
\textbf{0.19} & 41 & 52 & 70 & 34 & 41 & 42 \\ \hline
\end{tabularx}

El análisis realizado nos indica los siguientes resultados para sismos totales:

\begin{itemize}
\item Oaxaca (n=335) puede estimar la media con un error máximo de ±0.07, pues necesita 302 muestras

\item Guerrero (n=256) puede estimar la media con un error máximo de ±0.09, pues necesita 232 muestras

\item Michoacán (n=90) puede estimar la media con un error máximo de ±0.17, pues necesita 87 muestras

\item Chiapas (n=629) puede estimar la media con un error máximo de ±0.05, pues necesita 483 muestras

\item Resto Nacionales (n=545) puede estimar la media con un error máximo de ±0.06, pues necesita 403 muestras

\item Sismos Nacionales (n=1855) puede estimar la media con un error máximo de ±0.03, pues necesita de 1662 muestras
\end{itemize}

La tabla 12 muestra el tamaño mínimo de la muestra para las 6 regiones para magnitudes máximas.

%%\small
\footnotesize
\centering
\begin{tabularx}{15.9cm}{|c|* {6}{>{\centering\arraybackslash}X|}}
\caption{Tamaño mínimo de la muestra para las 6 regiones para magnitudes máximas.} \label{tab:tamano_muestra_maximas} \\
\hline
\rowcolor{gray!60}
\textbf{$\epsilon$} & \textbf{Oaxaca (n=64)} & \textbf{Guerrero (n=70)} & \textbf{Michoacán (n=50)} & \textbf{Chiapas (n=73)} & \textbf{Resto Nacionales (n=87)} & \textbf{Sismos Nacionales (n=110)} \\ \hline
\rowcolor{gray!20}
\textbf{0.09} & 306 & 310 & 423 & 269 & 164 & 125 \\ \hline
\textbf{0.1} & 248 & 252 & 342 & 218 & 133 & 101 \\ \hline
\rowcolor{gray!20}
\textbf{0.11} & 205 & 208 & 283 & 180 & 110 & 84 \\ \hline
\textbf{0.12} & 172 & 175 & 238 & 151 & 92 & 70 \\ \hline
\rowcolor{gray!20}
\textbf{0.13} & 147 & 149 & 203 & 129 & 79 & 60 \\ \hline
\textbf{0.14} & 127 & 129 & 175 & 111 & 68 & 52 \\ \hline
\rowcolor{gray!20}
\textbf{0.15} & 110 & 112 & 152 & 97 & 59 & 45 \\ \hline
\textbf{0.16} & 97 & 99 & 134 & 85 & 52 & 40 \\ \hline
\rowcolor{gray!20}
\textbf{0.17} & 86 & 87 & 119 & 76 & 46 & 35 \\ \hline
\textbf{0.18} & 77 & 78 & 106 & 68 & 41 & 32 \\ \hline
\rowcolor{gray!20}
\textbf{0.19} & 69 & 70 & 95 & 61 & 37 & 28 \\ \hline
\textbf{0.2} & 62 & 63 & 86 & 55 & 34 & 26 \\ \hline
\rowcolor{gray!20}
\textbf{0.21} & 57 & 57 & 78 & 50 & 31 & 23 \\ \hline
\textbf{0.22} & 52 & 52 & 71 & 45 & 28 & 21 \\ \hline
\rowcolor{gray!20}
\textbf{0.23} & 47 & 48 & 65 & 42 & 26 & 20 \\ \hline
\textbf{0.24} & 43 & 44 & 60 & 38 & 23 & 18 \\ \hline
\rowcolor{gray!20}
\textbf{0.25} & 40 & 41 & 55 & 35 & 22 & 17 \\ \hline
\textbf{0.26} & 37 & 38 & 51 & 33 & 20 & 15 \\ \hline
\rowcolor{gray!20}
\textbf{0.27} & 34 & 35 & 47 & 30 & 19 & 14 \\ \hline
\end{tabularx}

El análisis realizado nos indica los siguientes resultados para magnitudes máximas:

\begin{itemize}
\item Oaxaca (n=64) puede estimar la media con un error máximo de ±0.20, pues necesita 62 muestras

\item Guerrero (n=70) puede estimar la media con un error máximo de ±0.19, pues necesita 70 muestras

\item Michoacán (n=50) puede estimar la media con un error máximo de ±0.27, pues necesita 47 muestras

\item Chiapas (n=73) puede estimar la media con un error máximo de ±0.18, pues necesita 68 muestras

\item Resto Nacionales (n=87) puede estimar la media con un error máximo de ±0.13, pues necesita 79 muestras

\item Sismos Nacionales (n=110) puede estimar la media con un error máximo de ±0.10, pues necesita de 101 muestras
\end{itemize}
\section{Pruebas de hipótesis}

Las pruebas de hipótesis constituyen un procedimiento estadístico que permite tomar decisiones sobre parámetros de una población basándose en la evidencia muestral. Las pruebas de hipótesis involucran la formulación de dos hipótesis mutuamente excluyentes, que son:

\begin{itemize}
\item \textbf{Hipótesis nula (H0):} Es la afirmación que se presume verdadera inicialmente y que se desea contrastar.

\item \textbf{Hipótesis alternativa (H1):} Es la afirmación que se acepta si existe evidencia suficiente para rechazar la hipótesis nula.
\end{itemize}

\subsection{Prueba de hipótesis para el cociente de varianzas (prueba F)}

Se proponen las siguientes hipótesis:

\begin{align*}
\textbf{H0:} & \quad \sigma_1^2 = \sigma_2^2 \text{ (las varianzas son iguales)} \\
\textbf{H1:} & \quad \sigma_1^2 \neq \sigma_2^2 \text{ (las varianzas son diferentes)} \\
& \text{Nivel de significancia: } \alpha = 0.05
\end{align*}

La tabla 13 muestra los resultados de las pruebas de hipótesis realizadas para sismos totales.

\footnotesize
\centering
\begin{tabularx}{15.9cm}{|c|c|c|c|c|c|c|c|c|c|}
\caption{Resultados de las pruebas de hipótesis para cociente de varianzas para sismos totales.} \label{tab:hipotesis_varianzas_totales_13} \\
\hline
\rowcolor{gray!60}
\textbf{Estado 1} & \textbf{Estado 2} & \textbf{Var\_1} & \textbf{Var\_2} & \textbf{F\_est} & \textbf{GL\_num} & \textbf{GL\_den} & \textbf{P\_val} & \textbf{Sig.} & \textbf{Interpretación} \\ \hline
\rowcolor{gray!20}
Oaxaca & Guerrero & 0.3819 & 0.4834 & 1.2656 & 255 & 334 & 0.0439 & Sí & Gro. mayor var. \\ \hline
Oaxaca & Michoacán & 0.3819 & 0.6357 & 1.6645 & 89 & 334 & 0.0014 & Sí & Mich. mayor var. \\ \hline
\rowcolor{gray!20}
Oaxaca & Chiapas & 0.3819 & 0.3129 & 1.2207 & 334 & 628 & 0.0348 & Sí & Oax. mayor var. \\ \hline
Oaxaca & Resto Nals. & 0.3819 & 0.3751 & 1.0182 & 334 & 544 & 0.8485 & No & Sin dif. \\ \hline
\rowcolor{gray!20}
Oaxaca & Nacionales & 0.3819 & 0.3887 & 1.0177 & 1854 & 334 & 0.8498 & No & Sin dif. \\ \hline
Guerrero & Michoacán & 0.4834 & 0.6357 & 1.3152 & 89 & 255 & 0.1025 & No & Sin dif. \\ \hline
\rowcolor{gray!20}
Guerrero & Chiapas & 0.4834 & 0.3129 & 1.5449 & 255 & 628 & 0.0000 & Sí & Gro. mayor var. \\ \hline
Guerrero & Resto Nals. & 0.4834 & 0.3751 & 1.2885 & 255 & 544 & 0.0160 & Sí & Gro. mayor var. \\ \hline
\rowcolor{gray!20}
Guerrero & Nacionales & 0.4834 & 0.3887 & 1.2435 & 255 & 1854 & 0.0166 & Sí & Gro. mayor var. \\ \hline
Michoacán & Chiapas & 0.6357 & 0.3129 & 2.0318 & 89 & 628 & 0.0000 & Sí & Mich. mayor var. \\ \hline
\rowcolor{gray!20}
Michoacán & Resto Nals. & 0.6357 & 0.3751 & 1.6947 & 89 & 544 & 0.0004 & Sí & Mich. mayor var. \\ \hline
Michoacán & Nacionales & 0.6357 & 0.3887 & 1.6355 & 89 & 1854 & 0.0004 & Sí & Mich. mayor var. \\ \hline
\rowcolor{gray!20}
Chiapas & Resto Nals. & 0.3129 & 0.3751 & 1.1989 & 544 & 628 & 0.0281 & Sí & R.N. mayor var. \\ \hline
Chiapas & Nacionales & 0.3129 & 0.3887 & 1.2424 & 1854 & 628 & 0.0011 & Sí & Nals. mayor var. \\ \hline
\rowcolor{gray!20}
Resto Nals. & Nacionales & 0.3751 & 0.3887 & 1.0362 & 1854 & 544 & 0.6158 & No & Sin dif. \\ \hline
\end{tabularx}

Los resultados obtenidos indican lo siguiente:

\begin{itemize}
\item Se acepta la hipótesis nula (H0: $\sigma_1^2 = \sigma_2^2$ (las varianzas son iguales))

\begin{itemize}
\item Varianza Oaxaca = Resto Nacionales = Sismos Nacionales
\item Varianza Guerrero = Michoacán
\item Varianza Resto Nacionales = Sismos Nacionales
\end{itemize}

\item Se rechaza la hipótesis nula en favor de la alternativa (H1: $\sigma_1^2 \neq \sigma_2^2$ (las varianzas son diferentes))

\begin{itemize}
\item Varianza Oaxaca $\neq$ Guerrero $\neq$ Michoacán $\neq$ Chiapas
\item Varianza Guerrero $\neq$ Chiapas $\neq$ Resto Nacionales $\neq$ Sismos Nacional
\item Varianza Michoacán $\neq$ Chiapas $\neq$ Resto Nacionales $\neq$ Sismos Nacionales
\item Varianza Chiapas $\neq$ Resto Nacionales $\neq$ Sismos Nacionales
\end{itemize}
\end{itemize}

La tabla 14 muestra los resultados de las pruebas de hipótesis realizadas para magnitudes máximas.

\footnotesize
\centering
\begin{tabularx}{15.9cm}{|c|c|c|c|c|c|c|c|c|c|}
\caption{Resultados de las pruebas de hipótesis para cociente de varianzas para sismos totales.} \label{tab:hipotesis_varianzas_totales} \\
\hline
\rowcolor{gray!60}
\textbf{Estado 1} & \textbf{Estado 2} & \textbf{Var\_1} & \textbf{Var\_2} & \textbf{F\_est} & \textbf{GL\_num} & \textbf{GL\_den} & \textbf{P\_val} & \textbf{Sig.} & \textbf{Interpretación} \\ \hline
\rowcolor{gray!20}
Oaxaca & Guerrero & 0.6192 & 0.6309 & 1.0189 & 69 & 63 & 0.9427 & No & Sin dif. \\ \hline
Oaxaca & Michoacán & 0.6192 & 0.8468 & 1.3676 & 49 & 63 & 0.2406 & No & Sin dif. \\ \hline
\rowcolor{gray!20}
Oaxaca & Chiapas & 0.6192 & 0.5466 & 1.1327 & 63 & 72 & 0.6066 & No & Sin dif. \\ \hline
Oaxaca & Resto Nals. & 0.6192 & 0.3352 & 1.8474 & 63 & 86 & 0.0082 & Sí & Oax. mayor var. \\ \hline
\rowcolor{gray!20}
Oaxaca & Nacionales & 0.6192 & 0.2559 & 2.4196 & 63 & 109 & 0.0000 & Sí & Oax. mayor var. \\ \hline
Guerrero & Michoacán & 0.6309 & 0.8468 & 1.3422 & 49 & 69 & 0.2578 & No & Sin dif. \\ \hline
\rowcolor{gray!20}
Guerrero & Chiapas & 0.6309 & 0.5466 & 1.1541 & 69 & 72 & 0.5477 & No & Sin dif. \\ \hline
Guerrero & Resto Nals. & 0.6309 & 0.3352 & 1.8823 & 69 & 86 & 0.0055 & Sí & Gro. mayor var. \\ \hline
\rowcolor{gray!20}
Guerrero & Nacionales & 0.6309 & 0.2559 & 2.4653 & 69 & 109 & 0.0000 & Sí & Gro. mayor var. \\ \hline
Michoacán & Chiapas & 0.8468 & 0.5466 & 1.5491 & 49 & 72 & 0.0896 & No & Sin dif. \\ \hline
\rowcolor{gray!20}
Michoacán & Resto Nals. & 0.8468 & 0.3352 & 2.5265 & 49 & 86 & 0.0001 & Sí & Mich. mayor var. \\ \hline
Michoacán & Nacionales & 0.8468 & 0.2559 & 3.3089 & 49 & 109 & 0.0000 & Sí & Mich. mayor var. \\ \hline
\rowcolor{gray!20}
Chiapas & Resto Nals. & 0.5466 & 0.3352 & 1.6310 & 72 & 86 & 0.0299 & Sí & Chis. mayor var. \\ \hline
Chiapas & Nacionales & 0.5466 & 0.2559 & 2.1361 & 72 & 109 & 0.0003 & Sí & Chis. mayor var. \\ \hline
\rowcolor{gray!20}
Resto Nals. & Nacionales & 0.3352 & 0.2559 & 1.3097 & 86 & 109 & 0.1827 & No & Sin dif. \\ \hline
\end{tabularx}

Los resultados obtenidos indican lo siguiente:

\begin{itemize}
\item Se acepta la hipótesis nula (H0: $\sigma_1^2 = \sigma_2^2$ (las varianzas son iguales))

\begin{itemize}
\item Varianza Oaxaca = Guerrero = Michoacán = Chiapas
\item Varianza Guerrero = Michoacán = Chiapas
\item Varianza Michoacán = Chiapas
\item Varianza Resto Nacionales = Sismos Nacionales
\end{itemize}

\item Se rechaza la hipótesis nula en favor de la alternativa (H1: $\sigma_1^2 \neq \sigma_2^2$ (las varianzas son diferentes))

\begin{itemize}
\item Varianza Oaxaca $\neq$ Resto Nacionales $\neq$ Sismos Nacionales
\item Varianza Guerrero $\neq$ Resto Nacionales $\neq$ Sismos Nacional
\item Varianza Michoacán $\neq$ Resto Nacionales $\neq$ Sismos Nacionales
\item Varianza Chiapas $\neq$ Resto Nacionales $\neq$ Sismos Nacionales
\end{itemize}
\end{itemize}

\subsection{Prueba de hipótesis para diferencia de medias}

Se proponen las siguientes hipótesis:

\begin{align*}
\textbf{H0:} & \quad \mu_1^2 = \mu_2^2 \text{ (las medias son iguales)} \\
\textbf{H1:} & \quad \mu_1^2 \neq \mu_2^2 \text{ (las medias son diferentes)} \\
& \text{Nivel de significancia: } \alpha = 0.05
\end{align*}

La tabla 15 muestra los resultados de las pruebas de hipótesis realizadas para sismos totales.

\scriptsize
\centering
\begin{tabularx}{15.9cm}{|p{1.2cm}|p{1.2cm}|c|c|c|c|c|c|c|c|X|}
\caption{Resultados de las pruebas de hipótesis para diferencia de medias para sismos totales.} \label{tab:hipotesis_medias_totales} \\
\hline
\rowcolor{gray!60}
\textbf{Estado 1} & \textbf{Estado 2} & \textbf{Media 1} & \textbf{Media 2} & \textbf{Dif.} & \textbf{T\_est} & \textbf{GL} & \textbf{P\_val} & \textbf{Sig.} & \textbf{Prueba} & \textbf{Interpretación} \\ \hline
\rowcolor{gray!20}
Oaxaca & Guerrero & 5.4451 & 5.5734 & -0.1284 & -2.3326 & 513.10 & 0.0201 & Sí & Welch & Media Gro. $>$ Oax. \\ \hline
Oaxaca & Michoacán & 5.4451 & 5.6100 & -0.1649 & -1.8209 & 119.22 & 0.0711 & No & Welch & Sin diferencia significativa \\ \hline
\rowcolor{gray!20}
Oaxaca & Chiapas & 5.4451 & 5.3951 & 0.0500 & 1.2357 & 625.68 & 0.2171 & No & Welch & Sin diferencia significativa \\ \hline
Oaxaca & Resto Nals. & 5.4451 & 5.5677 & -0.1226 & -2.8741 & 878.00 & 0.0042 & Sí & Pooled & Media R.N. $>$ Oax. \\ \hline
\rowcolor{gray!20}
Oaxaca & Nacionales & 5.4451 & 5.4899 & -0.0448 & -1.2118 & 2188.00 & 0.2257 & No & Pooled & Sin diferencia significativa \\ \hline
Guerrero & Michoacán & 5.5734 & 5.6100 & -0.0366 & -0.4126 & 344.00 & 0.6801 & No & Pooled & Sin diferencia significativa \\ \hline
\rowcolor{gray!20}
Guerrero & Chiapas & 5.5734 & 5.3951 & 0.1784 & 3.6518 & 395.90 & 0.0003 & Sí & Welch & Media Gro. $>$ Chis. \\ \hline
Guerrero & Resto Nals. & 5.5734 & 5.5677 & 0.0057 & 0.1129 & 446.96 & 0.9102 & No & Welch & Sin diferencia significativa \\ \hline
\rowcolor{gray!20}
Guerrero & Nacionales & 5.5734 & 5.4899 & 0.0836 & 1.8247 & 314.21 & 0.0690 & No & Welch & Sin diferencia significativa \\ \hline
Michoacán & Chiapas & 5.6100 & 5.3951 & 0.2149 & 2.4717 & 101.90 & 0.0151 & Sí & Welch & Media Mich. $>$ Chis. \\ \hline
\rowcolor{gray!20}
Michoacán & Resto Nals. & 5.6100 & 5.5677 & 0.0423 & 0.4804 & 107.02 & 0.6319 & No & Welch & Sin diferencia significativa \\ \hline
Michoacán & Nacionales & 5.6100 & 5.4899 & 0.1201 & 1.4086 & 94.35 & 0.1622 & No & Welch & Sin diferencia significativa \\ \hline
\rowcolor{gray!20}
Chiapas & Resto Nals. & 5.3951 & 5.5677 & -0.1726 & -5.0134 & 1111.53 & 0.0000 & Sí & Welch & Media R.N. $>$ Chis. \\ \hline
Chiapas & Nacionales & 5.3951 & 5.4899 & -0.0948 & -3.5651 & 1196.62 & 0.0004 & Sí & Welch & Media Nals. $>$ Chis. \\ \hline
\rowcolor{gray!20}
Resto Nals. & Nacionales & 5.5677 & 5.4899 & 0.0778 & 2.5727 & 2398.00 & 0.0102 & Sí & Pooled & Media R.N. $>$ Nals. \\ \hline
\end{tabularx}

Los resultados obtenidos indican lo siguiente:

\begin{itemize}
\item Se acepta la hipótesis nula (H0: $\mu_1^2 = \mu_2^2$ (las medias son iguales))

\begin{itemize}
\item Media Oaxaca = Michoacán = Chiapas = Sismos Nacionales
\item Media Guerrero = Michoacán = Resto Nacionales = Sismos Nacionales
\item Media Michoacán = Resto Nacionales =Sismos Nacionales
\end{itemize}

\item Se rechaza la hipótesis nula en favor de la alternativa (H1: $\mu_1^2 \neq \mu_2^2$ (las medias son diferentes))

\begin{itemize}
\item Media Oaxaca $\neq$ Guerrero $\neq$ Resto Nacionales
\item Media Guerrero $\neq$ Chiapas
\item Media Michoacán $\neq$ Chiapas
\item Media Chiapas $\neq$ Resto Nacionales $\neq$ Sismos Nacionales
\item Media Resto Nacionales $\neq$ Sismos Nacionales
\end{itemize}
\end{itemize}

La tabla 16 muestra los resultados de las pruebas de hipótesis realizadas para magnitudes máximas.

\scriptsize
\centering
\begin{tabularx}{15.9cm}{|p{1.2cm}|p{1.2cm}|c|c|c|c|c|c|c|c|X|}
\caption{Resultados de las pruebas de hipótesis para diferencia de medias para magnitudes máximas.} \label{tab:hipotesis_medias_maximas} \\
\hline
\rowcolor{gray!60}
\textbf{Estado 1} & \textbf{Estado 2} & \textbf{Media 1} & \textbf{Media 2} & \textbf{Dif.} & \textbf{T\_est} & \textbf{GL} & \textbf{P\_val} & \textbf{Sig.} & \textbf{Prueba} & \textbf{Interpretación} \\ \hline
\rowcolor{gray!20}
Oaxaca & Guerrero & 6.2219 & 6.2886 & -0.0667 & -0.4877 & 132.00 & 0.6266 & No & Pooled & Sin diferencia significativa \\ \hline
Oaxaca & Michoacán & 6.2219 & 5.8880 & 0.3339 & 2.0865 & 112.00 & 0.0392 & Sí & Pooled & Media Oax. $>$ Mich. \\ \hline
\rowcolor{gray!20}
Oaxaca & Chiapas & 6.2219 & 6.4425 & -0.2206 & -1.6907 & 135.00 & 0.0932 & No & Pooled & Sin diferencia significativa \\ \hline
Oaxaca & Resto Nals. & 6.2219 & 6.5172 & -0.2954 & -2.5395 & 110.34 & 0.0125 & Sí & Welch & Media R.N. $>$ Oax. \\ \hline
\rowcolor{gray!20}
Oaxaca & Nacionales & 6.2219 & 7.0073 & -0.7854 & -7.1692 & 93.81 & 0.0000 & Sí & Welch & Media Nals. $>$ Oax. \\ \hline
Guerrero & Michoacán & 6.2886 & 5.8880 & 0.4006 & 2.5486 & 118.00 & 0.0121 & Sí & Pooled & Media Gro. $>$ Mich. \\ \hline
\rowcolor{gray!20}
Guerrero & Chiapas & 6.2886 & 6.4425 & -0.1539 & -1.1998 & 141.00 & 0.2322 & No & Pooled & Sin diferencia significativa \\ \hline
Guerrero & Resto Nals. & 6.2886 & 6.5172 & -0.2287 & -2.0161 & 122.62 & 0.0460 & Sí & Welch & Media R.N. $>$ Gro. \\ \hline
\rowcolor{gray!20}
Guerrero & Nacionales & 6.2886 & 7.0073 & -0.7187 & -6.7493 & 104.80 & 0.0000 & Sí & Welch & Media Nals. $>$ Gro. \\ \hline
Michoacán & Chiapas & 5.8880 & 6.4425 & -0.5545 & -3.6950 & 121.00 & 0.0003 & Sí & Pooled & Media Chis. $>$ Mich. \\ \hline
\rowcolor{gray!20}
Michoacán & Resto Nals. & 5.8880 & 6.5172 & -0.6292 & -4.3642 & 71.71 & 0.0000 & Sí & Welch & Media R.N. $>$ Mich. \\ \hline
Michoacán & Nacionales & 5.8880 & 7.0073 & -1.1193 & -8.0646 & 62.85 & 0.0000 & Sí & Welch & Media Nals. $>$ Mich. \\ \hline
\rowcolor{gray!20}
Chiapas & Resto Nals. & 6.4425 & 6.5172 & -0.0748 & -0.7022 & 135.18 & 0.4838 & No & Welch & Sin diferencia significativa \\ \hline
Chiapas & Nacionales & 6.4425 & 7.0073 & -0.5648 & -5.7011 & 116.27 & 0.0000 & Sí & Welch & Media Nals. $>$ Chis. \\ \hline
\rowcolor{gray!20}
Resto Nals. & Nacionales & 6.5172 & 7.0073 & -0.4900 & -6.3329 & 195.00 & 0.0000 & Sí & Pooled & Media Nals. $>$ R.N. \\ \hline
\end{tabularx}

Los resultados obtenidos indican lo siguiente:

\begin{itemize}
\item Se acepta la hipótesis nula (H0: $\mu_1^2 = \mu_2^2$ (las medias son iguales))

\begin{itemize}
\item Media Oaxaca = Guerrero = Chiapas
\item Media Guerrero = Chiapas
\item Media Chiapas = Resto Nacionales
\end{itemize}

\item Se rechaza la hipótesis nula en favor de la alternativa (H1: $\mu_1^2 \neq \mu_2^2$ (las medias son diferentes))

\begin{itemize}
\item Media Oaxaca $\neq$ Michoacán $\neq$ Resto Nacionales $\neq$ Sismos Nacionales
\item Media Guerrero $\neq$ Michoacán $\neq$ Resto Nacionales $\neq$ Sismos Nacional
\item Media Michoacán $\neq$ Chiapas $\neq$ Resto Nacionales $\neq$ Sismos Nacionales
\item Media Chiapas $\neq$ Sismos Nacionales
\item Media Resto Nacionales $\neq$ Sismos Nacionales
\end{itemize}
\end{itemize}
\subsection{Prueba de hipótesis para diferencia de proporciones iguales}

Se proponen las siguientes hipótesis:

\begin{align*}
\textbf{H0:} & \quad p_1^2 = p_2^2 \text{ (las proporciones son iguales)} \\
\textbf{H1:} & \quad p_1^2 \neq p_2^2 \text{ (las proporciones son diferentes)} \\
& \text{Nivel de significancia: } \alpha = 0.05
\end{align*}

La tabla 17 muestra los resultados de las pruebas de hipótesis realizadas para sismos totales.

\footnotesize
\centering
\begin{tabularx}{15.9cm}{|p{1.4cm}|p{1.4cm}|c|c|c|c|c|c|X|}
\caption{Resultados de las pruebas de hipótesis para proporciones iguales para sismos totales.} \label{tab:hipotesis_proporciones_totales} \\
\hline
\rowcolor{gray!60}
\textbf{Estado 1} & \textbf{Estado 2} & \textbf{Prop\_1} & \textbf{Prop\_2} & \textbf{Dif.} & \textbf{Z\_est} & \textbf{P\_val} & \textbf{Sig.} & \textbf{Interpretación} \\ \hline
\rowcolor{gray!20}
Oaxaca & Guerrero & 0.0896 & 0.1445 & -0.0549 & -2.0860 & 0.0369 & Sí & Proporción Gro. $>$ Oax. \\ \hline
Oaxaca & Michoacán & 0.0896 & 0.1667 & -0.0771 & -2.1105 & 0.0348 & Sí & Proporción Mich. $>$ Oax. \\ \hline
\rowcolor{gray!20}
Oaxaca & Chiapas & 0.0896 & 0.0700 & 0.0196 & 1.0885 & 0.2763 & No & Sin diferencia significativa \\ \hline
Oaxaca & Resto Nals. & 0.0896 & 0.1009 & -0.0113 & -0.5510 & 0.5816 & No & Sin diferencia significativa \\ \hline
\rowcolor{gray!20}
Oaxaca & Nacionales & 0.0896 & 0.0976 & -0.0080 & -0.4567 & 0.6478 & No & Sin diferencia significativa \\ \hline
Guerrero & Michoacán & 0.1445 & 0.1667 & -0.0222 & -0.5069 & 0.6121 & No & Sin diferencia significativa \\ \hline
\rowcolor{gray!20}
Guerrero & Chiapas & 0.1445 & 0.0700 & 0.0745 & 3.4850 & 0.0004 & Sí & Proporción Gro. $>$ Chis. \\ \hline
Guerrero & Resto Nals. & 0.1445 & 0.1009 & 0.0436 & 1.8047 & 0.0711 & No & Sin diferencia sig. \\ \hline
\rowcolor{gray!20}
Guerrero & Nacionales & 0.1445 & 0.0976 & 0.0469 & 2.3116 & 0.0208 & Sí & Proporción Gro. $>$ Nals. \\ \hline
Michoacán & Chiapas & 0.1667 & 0.0700 & 0.0967 & 3.1264 & 0.0017 & Sí & Proporción Mich. $>$ Chis. \\ \hline
\rowcolor{gray!20}
Michoacán & Resto Nals. & 0.1667 & 0.1009 & 0.0658 & 1.8465 & 0.0648 & No & Sin diferencia sig. \\ \hline
Michoacán & Nacionales & 0.1667 & 0.0976 & 0.0691 & 2.1267 & 0.0334 & Sí & Proporción Mich. $>$ Nals. \\ \hline
\rowcolor{gray!20}
Chiapas & Resto Nals. & 0.0700 & 0.1009 & -0.0309 & -1.9002 & 0.0574 & No & Sin diferencia sig. \\ \hline
Chiapas & Nacionales & 0.0700 & 0.0976 & -0.0276 & -2.0842 & 0.0371 & Sí & Proporción Nals. $>$ Chis. \\ \hline
\rowcolor{gray!20}
Resto Nals. & Nacionales & 0.1009 & 0.0976 & 0.0033 & 0.2275 & 0.8200 & No & Sin diferencia sig. \\ \hline
\end{tabularx}

Los resultados obtenidos indican lo siguiente:

\begin{itemize}
\item Se acepta la hipótesis nula (H0: $p_1^2 = p_2^2$ (las proporciones son iguales))

\begin{itemize}
\item Proporción Oaxaca = Chiapas = Resto Nacionales = Sismos Nacionales
\item Proporción Guerrero = Michoacán = Resto Nacionales
\item Proporción Michoacán = Sismos Nacionales
\item Proporción Chiapas = Resto Nacionales
\item Proporción Resto Nacionales = Sismos Nacionales
\end{itemize}

\item Se rechaza la hipótesis nula en favor de la alternativa (H1: $p_1^2 \neq p_2^2$ (las proporciones son diferentes))

\begin{itemize}
\item Proporción Oaxaca $\neq$ Guerrero $\neq$ Michoacán
\item Proporción Guerrero $\neq$ Chiapas $\neq$ Sismos Nacionales
\item Proporción Michoacán $\neq$ Chiapas $\neq$ Sismos Nacionales
\item Proporción Chiapas $\neq$ Sismos Nacionales
\end{itemize}
\end{itemize}

La tabla 18 muestra los resultados de las pruebas de hipótesis realizadas para magnitudes máximas.

\footnotesize
\centering
\begin{tabularx}{15.9cm}{|p{1.4cm}|p{1.4cm}|c|c|c|c|c|c|X|}
\caption{Resultados de las pruebas de hipótesis para proporciones iguales para magnitudes máximas.} \label{tab:hipotesis_proporciones_maximas} \\
\hline
\rowcolor{gray!60}
\textbf{Estado 1} & \textbf{Estado 2} & \textbf{Prop 1} & \textbf{Prop 2} & \textbf{Dif.} & \textbf{Z\_est} & \textbf{P\_valor} & \textbf{Sig.} & \textbf{Interpretación} \\ \hline
\rowcolor{gray!20}
Oaxaca & Guerrero & 0.3750 & 0.4429 & -0.0679 & -0.7981 & 0.4248 & No & Sin diferencia significativa \\ \hline
Oaxaca & Michoacán & 0.3750 & 0.2800 & 0.0950 & 1.0877 & 0.2857 & No & Sin diferencia significativa \\ \hline
\rowcolor{gray!20}
Oaxaca & Chiapas & 0.3750 & 0.4932 & -0.1182 & -1.3913 & 0.1641 & No & Sin diferencia significativa \\ \hline
Oaxaca & Resto Nals. & 0.3750 & 0.4828 & -0.1078 & -1.3197 & 0.1869 & No & Sin diferencia significativa \\ \hline
\rowcolor{gray!20}
Oaxaca & Nacionales & 0.3750 & 0.8182 & -0.4432 & -5.9311 & 0.0000 & Sí & Proporción Nals. $>$ Oax. \\ \hline
Guerrero & Michoacán & 0.4429 & 0.2800 & 0.1629 & 1.8172 & 0.0692 & No & Sin diferencia significativa \\ \hline
\rowcolor{gray!20}
Guerrero & Chiapas & 0.4429 & 0.4932 & -0.0503 & -0.6026 & 0.5468 & No & Sin diferencia significativa \\ \hline
Guerrero & Resto Nals. & 0.4429 & 0.4828 & -0.0399 & -0.4982 & 0.6183 & No & Sin diferencia significativa \\ \hline
\rowcolor{gray!20}
Guerrero & Nacionales & 0.4429 & 0.8182 & -0.3753 & -5.2293 & 0.0000 & Sí & Proporción Nals. $>$ Gro. \\ \hline
Michoacán & Chiapas & 0.2800 & 0.4932 & -0.2132 & -2.3645 & 0.0181 & Sí & Proporción Chis. $>$ Mich. \\ \hline
\rowcolor{gray!20}
Michoacán & Resto Nals. & 0.2800 & 0.4828 & -0.2028 & -2.3245 & 0.0201 & Sí & Proporción R.N. $>$ Mich. \\ \hline
Michoacán & Nacionales & 0.2800 & 0.8182 & -0.5382 & -6.6157 & 0.0000 & Sí & Proporción Nals. $>$ Mich. \\ \hline
\rowcolor{gray!20}
Chiapas & Resto Nals. & 0.4932 & 0.4828 & 0.0104 & 0.1311 & 0.8957 & No & Sin diferencia significativa \\ \hline
Chiapas & Nacionales & 0.4932 & 0.8182 & -0.3250 & -4.6488 & 0.0000 & Sí & Proporción Nals. $>$ Chis. \\ \hline
\rowcolor{gray!20}
Resto Nals. & Nacionales & 0.4828 & 0.8182 & -0.3354 & -4.9717 & 0.0000 & Sí & Proporción Nals. $>$ R.N. \\ \hline
\end{tabularx}

Los resultados obtenidos indican lo siguiente:

\begin{itemize}
\item Se acepta la hipótesis nula (H0: $p_1^2 = p_2^2$ (las proporciones son iguales))

\begin{itemize}
\item Proporción Oaxaca = Guerrero = Michoacán = Chiapas = Resto Nacionales
\item Proporción Guerrero = Michoacán = Chiapas = Resto Nacionales
\item Proporción Chiapas = Resto Nacionales
\end{itemize}

\item Se rechaza la hipótesis nula en favor de la alternativa (H1: $p_1^2 \neq p_2^2$ (las proporciones son diferentes))

\begin{itemize}
\item Proporción Oaxaca $\neq$ Sismos Nacionales
\item Proporción Guerrero $\neq$ Sismos Nacionales
\item Proporción Michoacán $\neq$ Chiapas $\neq$ Resto Nacionales $\neq$ Sismos Nacionales
\item Proporción Chiapas $\neq$ Sismos Nacionales
\item Proporción Resto Nacionales $\neq$ Sismos Nacionales
\end{itemize}
\end{itemize}

\subsection{Prueba de hipótesis para diferencias de proporciones diferentes}

Se proponen las siguientes hipótesis:

\begin{align*}
\textbf{H0:} & \quad p_1^2 = p_2^2 \text{ (las proporciones son iguales)} \\
\textbf{H1:} & \quad p_1^2 \neq p_2^2 \text{ (las proporciones son diferentes)} \\
& \text{Nivel de significancia: } \alpha = 0.05
\end{align*}

La tabla 19 muestra los resultados de las pruebas de hipótesis realizadas para sismos totales.

%%\scriptsize
\footnotesize
\centering
\begin{tabularx}{15.9cm}{|p{1.4cm}|p{1.4cm}|c|c|c|c|c|c|X|}
\caption{Resultados de las pruebas de hipótesis para proporciones diferentes para sismos totales.} \label{tab:hipotesis_proporciones_dif_totales} \\
\hline
\rowcolor{gray!60}
\textbf{Estado 1} & \textbf{Estado 2} & \textbf{Prop 1} & \textbf{Prop 2} & \textbf{Dif.} & \textbf{Z\_est} & \textbf{P\_valor} & \textbf{Sig.} & \textbf{Interpretación} \\ \hline
\rowcolor{gray!20}
Oaxaca & Guerrero & 0.0896 & 0.1445 & -0.0549 & -2.0370 & 0.0417 & Sí & Proporción Gro. $>$ Oax. \\ \hline
Oaxaca & Michoacán & 0.0896 & 0.1667 & -0.0771 & -1.8239 & 0.0682 & No & Sin diferencia significativa \\ \hline
\rowcolor{gray!20}
Oaxaca & Chiapas & 0.0896 & 0.0700 & 0.0196 & 1.0522 & 0.2927 & No & Sin diferencia significativa \\ \hline
Oaxaca & Resto Nals. & 0.0896 & 0.1009 & -0.0113 & -0.5581 & 0.5768 & No & Sin diferencia significativa \\ \hline
\rowcolor{gray!20}
Oaxaca & Nacionales & 0.0896 & 0.0976 & -0.0080 & -0.4690 & 0.6391 & No & Sin diferencia significativa \\ \hline
Guerrero & Michoacán & 0.1445 & 0.1667 & -0.0222 & -0.4932 & 0.6219 & No & Sin diferencia significativa \\ \hline
\rowcolor{gray!20}
Guerrero & Chiapas & 0.1445 & 0.0700 & 0.0745 & 3.0766 & 0.0021 & Sí & Proporción Gro. $>$ Chis. \\ \hline
Guerrero & Resto Nals. & 0.1445 & 0.1009 & 0.0436 & 1.7110 & 0.0871 & No & Sin diferencia significativa \\ \hline
\rowcolor{gray!20}
Guerrero & Nacionales & 0.1445 & 0.0976 & 0.0469 & 2.0365 & 0.0417 & Sí & Prop. Guerrero $>$ Nals. \\ \hline
Michoacán & Chiapas & 0.1667 & 0.0700 & 0.0967 & 2.3828 & 0.0172 & Sí & Proporción Mich. $>$ Chis. \\ \hline
\rowcolor{gray!20}
Michoacán & Resto Nals. & 0.1667 & 0.1009 & 0.0658 & 1.5913 & 0.1116 & No & Sin diferencia significativa \\ \hline
Michoacán & Nacionales & 0.1667 & 0.0976 & 0.0691 & 1.7324 & 0.0832 & No & Sin diferencia significativa \\ \hline
\rowcolor{gray!20}
Chiapas & Resto Nals. & 0.0700 & 0.1009 & -0.0309 & -1.8807 & 0.0600 & No & Sin diferencia significativa \\ \hline
Chiapas & Nacionales & 0.0700 & 0.0976 & -0.0276 & -2.2462 & 0.0247 & Sí & Proporción Nals. $>$ Chis. \\ \hline
\rowcolor{gray!20}
Resto Nals. & Nacionales & 0.1009 & 0.0976 & 0.0033 & 0.2256 & 0.8215 & No & Sin diferencia significativa \\ \hline
\end{tabularx}

Los resultados obtenidos indican lo siguiente:

\begin{itemize}
\item Se acepta la hipótesis nula (H0: $p_1^2 = p_2^2$ (las proporciones son iguales))

\begin{itemize}
\item Proporción Oaxaca = Michoacán = Chiapas = Resto Nacionales = Sismos Nacionales
\item Proporción Guerrero = Michoacán = Resto Nacionales
\item Proporción Michoacán = Resto Nacionales = Sismos Nacionales
\item Proporción Chiapas = Resto Nacionales
\item Proporción Resto Nacionales = Sismos Nacionales
\end{itemize}

\item Se rechaza la hipótesis nula en favor de la alternativa (H1: $p_1^2 \neq p_2^2$ (las proporciones son diferentes))

\begin{itemize}
\item Proporción Oaxaca $\neq$ Guerrero
\item Proporción Guerrero $\neq$ Chiapas $\neq$ Sismos Nacionales
\item Proporción Michoacán $\neq$ Chiapas
\item Proporción Chiapas $\neq$ Sismos Nacionales
\end{itemize}
\end{itemize}

La tabla 20 muestra los resultados de las pruebas de hipótesis realizadas para magnitudes máximas.

%%\scriptsize
\footnotesize
\centering
\begin{tabularx}{15.9cm}{|p{1.4cm}|p{1.4cm}|c|c|c|c|c|c|X|}
\caption{Resultados de las pruebas de hipótesis para proporciones diferentes para magnitudes máximas.} \label{tab:hipotesis_proporciones_dif_maximas} \\
\hline
\rowcolor{gray!60}
\textbf{Estado 1} & \textbf{Estado 2} & \textbf{Prop 1} & \textbf{Prop 2} & \textbf{Dif.} & \textbf{Z\_est} & \textbf{P\_valor} & \textbf{Sig.} & \textbf{Interpretación} \\ \hline
\rowcolor{gray!20}
Oaxaca & Guerrero & 0.3750 & 0.4429 & -0.0679 & -0.8009 & 0.4232 & No & Sin diferencia significativa \\ \hline
Oaxaca & Michoacán & 0.3750 & 0.2800 & 0.0950 & 1.0830 & 0.2788 & No & Sin diferencia significativa \\ \hline
\rowcolor{gray!20}
Oaxaca & Chiapas & 0.3750 & 0.4932 & -0.1182 & -1.4041 & 0.1603 & No & Sin diferencia significativa \\ \hline
Oaxaca & Resto Nals. & 0.3750 & 0.4828 & -0.1078 & -1.3338 & 0.1823 & No & Sin diferencia significativa \\ \hline
\rowcolor{gray!20}
Oaxaca & Nacionales & 0.3750 & 0.8182 & -0.4432 & -6.2588 & 0.0000 & Sí & Proporción Nals. $>$ Oax. \\ \hline
Guerrero & Michoacán & 0.4429 & 0.2800 & 0.1629 & 1.8739 & 0.0609 & No & Sin diferencia significativa \\ \hline
\rowcolor{gray!20}
Guerrero & Chiapas & 0.4429 & 0.4932 & -0.0503 & -0.6034 & 0.5462 & No & Sin diferencia significativa \\ \hline
Guerrero & Resto Nals. & 0.4429 & 0.4828 & -0.0399 & -0.4989 & 0.6178 & No & Sin diferencia significativa \\ \hline
\rowcolor{gray!20}
Guerrero & Nacionales & 0.4429 & 0.8182 & -0.3753 & -5.3740 & 0.0000 & Sí & Proporción Nals. $>$ Gro. \\ \hline
Michoacán & Chiapas & 0.2800 & 0.4932 & -0.2132 & -2.4691 & 0.0135 & Sí & Proporción Chis. $>$ Mich. \\ \hline
\rowcolor{gray!20}
Michoacán & Resto Nals. & 0.2800 & 0.4828 & -0.2028 & -2.4410 & 0.0146 & Sí & Prop. R.N. $>$ Mich. \\ \hline
Michoacán & Nacionales & 0.2800 & 0.8182 & -0.5382 & -7.3347 & 0.0000 & Sí & Proporción Nals. $>$ Mich. \\ \hline
\rowcolor{gray!20}
Chiapas & Resto Nals. & 0.4932 & 0.4828 & 0.0104 & 0.1311 & 0.8957 & No & Sin diferencia significativa \\ \hline
Chiapas & Nacionales & 0.4932 & 0.8182 & -0.3250 & -4.7026 & 0.0000 & Sí & Proporción Nals. $>$ Chis. \\ \hline
\rowcolor{gray!20}
Resto Nals. & Nacionales & 0.4828 & 0.8182 & -0.3354 & -5.1616 & 0.0000 & Sí & Proporción Nals. $>$ R.N. \\ \hline
\end{tabularx}

Los resultados obtenidos indican lo siguiente:

\begin{itemize}
\item Se acepta la hipótesis nula (H0: $p_1^2 = p_2^2$ (las proporciones son iguales))

\begin{itemize}
\item Proporción Oaxaca = Guerrero = Michoacán = Chiapas = Resto Nacionales
\item Proporción Guerrero = Michoacán = Chiapas = Resto Nacionales
\item Proporción Chiapas = Resto Nacionales
\end{itemize}

\item Se rechaza la hipótesis nula en favor de la alternativa (H1: $p_1^2 \neq p_2^2$ (las proporciones son diferentes))

\begin{itemize}
\item Proporción Oaxaca $\neq$ Sismos Nacionales
\item Proporción Guerrero $\neq$ Sismos Nacionales
\item Proporción Michoacán $\neq$ Chiapas $\neq$ Resto Nacionales $\neq$ Sismos Nacionales
\item Proporción Chiapas $\neq$ Sismos Nacionales
\item Proporción Restos Nacionales $\neq$ Sismos Nacionales
\end{itemize}
\end{itemize}
\subsection{Resumen de las pruebas de hipótesis}

A continuación, se presenta un resumen de todas las pruebas de hipótesis realizadas.

\begin{verbatim}
==============================================
RESUMEN DE LAS PRUEBAS DE HIPÓTESIS REALIZADAS
==============================================

1. PRUEBAS F - COCIENTE DE VARIANZAS:

 SISMOS TOTALES:
 • Oaxaca vs Guerrero: F=1.266, p=0.0440 - VARIANZAS DIFERENTES
 • Oaxaca vs Michoacán: F=1.665, p=0.0014 - VARIANZAS DIFERENTES
 • Oaxaca vs Chiapas: F=1.221, p=0.0348 - VARIANZAS DIFERENTES
 • Guerrero vs Chiapas: F=1.545, p=0.0000 - VARIANZAS DIFERENTES
 • Guerrero vs Resto Nacionales: F=1.288, p=0.0161 - VARIANZAS DIFERENTES
 • Guerrero vs Nacionales: F=1.244, p=0.0167 - VARIANZAS DIFERENTES
 • Michoacán vs Chiapas: F=2.032, p=0.0000 - VARIANZAS DIFERENTES
 • Michoacán vs Resto Nacionales: F=1.695, p=0.0005 - VARIANZAS DIFERENTES
 • Michoacán vs Nacionales: F=1.635, p=0.0005 - VARIANZAS DIFERENTES
 • Chiapas vs Resto Nacionales: F=1.199, p=0.0282 - VARIANZAS DIFERENTES
 • Chiapas vs Nacionales: F=1.242, p=0.0011 - VARIANZAS DIFERENTES

 MAGNITUDES MÁXIMAS:
 • Oaxaca vs Resto Nacionales: F=1.847, p=0.0083 - VARIANZAS DIFERENTES
 • Oaxaca vs Nacionales: F=2.420, p=0.0001 - VARIANZAS DIFERENTES
 • Guerrero vs Resto Nacionales: F=1.882, p=0.0055 - VARIANZAS DIFERENTES
 • Guerrero vs Nacionales: F=2.465, p=0.0000 - VARIANZAS DIFERENTES
 • Michoacán vs Resto Nacionales: F=2.526, p=0.0002 - VARIANZAS DIFERENTES
 • Michoacán vs Nacionales: F=3.309, p=0.0000 - VARIANZAS DIFERENTES
 • Chiapas vs Resto Nacionales: F=1.631, p=0.0300 - VARIANZAS DIFERENTES
 • Chiapas vs Nacionales: F=2.136, p=0.0003 - VARIANZAS DIFERENTES

2. PRUEBAS T - DIFERENCIA DE MEDIAS:

 SISMOS TOTALES:
 • Oaxaca vs Guerrero: t=-2.333, p=0.0201 - MEDIAS DIFERENTES (Varianzas diferentes (Welch))
 • Oaxaca vs Resto Nacionales: t=-2.874, p=0.0042 - MEDIAS DIFERENTES (Varianzas iguales (Pooled))
 • Guerrero vs Chiapas: t=3.652, p=0.0003 - MEDIAS DIFERENTES (Varianzas diferentes (Welch))
 • Michoacán vs Chiapas: t=2.472, p=0.0151 - MEDIAS DIFERENTES (Varianzas diferentes (Welch))
 • Chiapas vs Resto Nacionales: t=-5.013, p=0.0000 - MEDIAS DIFERENTES (Varianzas diferentes (Welch))
 • Chiapas vs Nacionales: t=-3.565, p=0.0004 - MEDIAS DIFERENTES (Varianzas diferentes (Welch))
 • Resto Nacionales vs Nacionales: t=2.573, p=0.0102 - MEDIAS DIFERENTES (Varianzas iguales (Pooled))

 MAGNITUDES MÁXIMAS:
 • Oaxaca vs Michoacán: t=2.087, p=0.0392 - MEDIAS DIFERENTES (Varianzas iguales (Pooled))
 • Oaxaca vs Resto Nacionales: t=-2.539, p=0.0125 - MEDIAS DIFERENTES (Varianzas diferentes (Welch))
 • Oaxaca vs Nacionales: t=-7.169, p=0.0000 - MEDIAS DIFERENTES (Varianzas diferentes (Welch))
 • Guerrero vs Michoacán: t=2.549, p=0.0121 - MEDIAS DIFERENTES (Varianzas iguales (Pooled))
 • Guerrero vs Resto Nacionales: t=-2.016, p=0.0460 - MEDIAS DIFERENTES (Varianzas diferentes (Welch))
 • Guerrero vs Nacionales: t=-6.749, p=0.0000 - MEDIAS DIFERENTES (Varianzas diferentes (Welch))
 • Michoacán vs Chiapas: t=-3.695, p=0.0003 - MEDIAS DIFERENTES (Varianzas iguales (Pooled))
 • Michoacán vs Resto Nacionales: t=-4.364, p=0.0000 - MEDIAS DIFERENTES (Varianzas diferentes (Welch))
 • Michoacán vs Nacionales: t=-8.065, p=0.0000 - MEDIAS DIFERENTES (Varianzas diferentes (Welch))
 • Chiapas vs Nacionales: t=-5.701, p=0.0000 - MEDIAS DIFERENTES (Varianzas diferentes (Welch))
 • Resto Nacionales vs Nacionales: t=-6.333, p=0.0000 - MEDIAS DIFERENTES (Varianzas iguales (Pooled))

3. PRUEBAS DE PROPORCIONES (Método Pooled):

 SISMOS TOTALES:
 • Oaxaca vs Guerrero: Z=-2.086, p=0.0370 - PROPORCIONES DIFERENTES
 • Oaxaca vs Michoacán: Z=-2.111, p=0.0348 - PROPORCIONES DIFERENTES
 • Guerrero vs Chiapas: Z=3.485, p=0.0005 - PROPORCIONES DIFERENTES
 • Guerrero vs Nacional: Z=2.312, p=0.0208 - PROPORCIONES DIFERENTES
 • Michoacán vs Chiapas: Z=3.126, p=0.0018 - PROPORCIONES DIFERENTES
 • Michoacán vs Nacional: Z=2.127, p=0.0334 - PROPORCIONES DIFERENTES
 • Chiapas vs Nacional: Z=-2.084, p=0.0371 - PROPORCIONES DIFERENTES

 MAGNITUDES MÁXIMAS:
 • Oaxaca vs Nacional: Z=-5.931, p=0.0000 - PROPORCIONES DIFERENTES
 • Guerrero vs Nacional: Z=-5.229, p=0.0000 - PROPORCIONES DIFERENTES
 • Michoacán vs Chiapas: Z=-2.365, p=0.0181 - PROPORCIONES DIFERENTES
 • Michoacán vs Resto Nacionales: Z=-2.325, p=0.0201 - PROPORCIONES DIFERENTES
 • Michoacán vs Nacional: Z=-6.616, p=0.0000 - PROPORCIONES DIFERENTES
 • Chiapas vs Nacional: Z=-4.649, p=0.0000 - PROPORCIONES DIFERENTES
 • Resto Nacionales vs Nacional: Z=-4.972, p=0.0000 - PROPORCIONES DIFERENTES

4. PRUEBAS DE PROPORCIONES (Sin Pooling):

 SISMOS TOTALES:
 • Oaxaca vs Guerrero: Z=-2.037, p=0.0417 - PROPORCIONES DIFERENTES
 • Guerrero vs Chiapas: Z=3.077, p=0.0021 - PROPORCIONES DIFERENTES
 • Guerrero vs Nacional: Z=2.037, p=0.0417 - PROPORCIONES DIFERENTES
 • Michoacán vs Chiapas: Z=2.383, p=0.0172 - PROPORCIONES DIFERENTES
 • Chiapas vs Nacional: Z=-2.246, p=0.0247 - PROPORCIONES DIFERENTES

 MAGNITUDES MÁXIMAS:
 • Oaxaca vs Nacional: Z=-6.259, p=0.0000 - PROPORCIONES DIFERENTES
 • Guerrero vs Nacional: Z=-5.374, p=0.0000 - PROPORCIONES DIFERENTES
 • Michoacán vs Chiapas: Z=-2.469, p=0.0135 - PROPORCIONES DIFERENTES
 • Michoacán vs Resto Nacionales: Z=-2.441, p=0.0146 - PROPORCIONES DIFERENTES
 • Michoacán vs Nacional: Z=-7.335, p=0.0000 - PROPORCIONES DIFERENTES
 • Chiapas vs Nacional: Z=-4.703, p=0.0000 - PROPORCIONES DIFERENTES
 • Resto Nacionales vs Nacional: Z=-5.162, p=0.0000 - PROPORCIONES DIFERENTES

==========================================
Nivel de significancia utilizado: α = 0.05
==========================================
\end{verbatim}

\section{Criterio de Información Bayesiano (BIC)}

Se probaron 20 distribuciones con BIC, y a continuación se presenta el resumen de las mejores distribuciones encontradas por estado para Sismos Totales y Magnitudes Máximas.

Mejor distribución por estado (menor BIC encontrado):

\begin{itemize}
\item Oaxaca: Totales=Pareto (117.610) Máximas=Uniforme (140.109)
\item Guerrero: Totales=Pareto (227.956) Máximas=Uniforme (152.644)
\item Michoacán: Totales=Pareto (94.176) Máximas=Pareto (95.798)
\item Chiapas: Totales=Pareto (63.043) Máximas=Normal (170.650)
\item Resto Nal.: Totales=Pareto (479.788) Máximas=Normal (159.719)
\item Nacional: Totales=Pareto (991.025) Máximas=Normal (170.641)
\end{itemize}

\section{Pruebas de bondad de ajuste KS, AD, LF}

A continuación, se presentan las tablas con los resultados completos de las PBA hechas a 20 distribuciones para cada región. Se incluye una tabla con los resultados para sismos totales y otra con los de magnitudes máximas anuales para cada región.

\subsection{Oaxaca}

%%\scriptsize
\footnotesize
\centering
\begin{tabularx}{15.9cm}{|l|c|c|c|c|c|c|X|}
\caption{Resultados de las pruebas de bondad de ajuste para sismos totales.} \label{tab:bondad_totales_p1} \\
\hline
\rowcolor{gray!60}
\textbf{Distribución} & \textbf{KS (D)} & \textbf{KS (P)} & \textbf{AD (An)} & \textbf{AD (P)} & \textbf{Lillie (D)} & \textbf{Lillie (P)} & \textbf{Interpretación} \\ \hline
\rowcolor{gray!20}
Gauss & 0.23537 & 1.51e-16 & 32.9044 & 1.79e-06 & 0.23571 & 5.62e-52 & Rechazo de ajuste \\ \hline
Gamma & 0.22551 & 3.18e-15 & 30.0427 & 1.79e-06 & 0.22601 & 1.26e-47 & Rechazo de ajuste \\ \hline
\rowcolor{gray!20}
Logística & 0.24940 & 1.58e-18 & 23.9971 & 1.79e-06 & 0.21817 & 3.00e-44 & Rechazo de ajuste \\ \hline
Laplace & 0.29922 & 1.77e-26 & 26.9683 & 1.79e-06 & 0.19297 & 3.03e-34 & Rechazo de ajuste \\ \hline
\rowcolor{gray!20}
Weibull & 0.30143 & 7.26e-27 & 39.5019 & 1.79e-06 & 0.25865 & 4.85e-63 & Rechazo de ajuste \\ \hline
Cauchy & 0.28337 & 8.62e-24 & 35.5641 & 1.79e-06 & 0.19767 & 5.17e-36 & Rechazo de ajuste \\ \hline
\rowcolor{gray!20}
Log Gauss & 0.22084 & 1.28e-14 & 28.6359 & 1.79e-06 & 0.22119 & 1.56e-45 & Rechazo de ajuste \\ \hline
Gauss Inversa & 0.22143 & 1.08e-14 & 28.7955 & 1.79e-06 & 0.22141 & 1.25e-45 & Rechazo de ajuste \\ \hline
\rowcolor{gray!20}
Log Logística & 0.23592 & 1.27e-16 & 21.3063 & 1.79e-06 & 0.20506 & 6.97e-39 & Rechazo de ajuste \\ \hline
Beta & 0.21778 & 3.16e-14 & 27.7415 & 1.79e-06 & 0.21792 & 3.84e-44 & Rechazo de ajuste \\ \hline
\rowcolor{gray!20}
Uniforme & 0.56481 & 2.97e-93 & Inf & 1.79e-06 & 0.35567 & 9.92e-122 & Rechazo de ajuste \\ \hline
Potencia & 0.40057 & 4.08e-47 & 86.5364 & 1.79e-06 & 0.30374 & 6.45e-88 & Rechazo de ajuste \\ \hline
\rowcolor{gray!20}
Pareto & 0.99983 & 2.62e-291 & 2645.73 & 1.79e-06 & 0.22226 & 5.40e-46 & Rechazo de ajuste \\ \hline
Rayleigh & 0.56504 & 2.50e-93 & 104.733 & 1.79e-06 & 0.23091 & 8.45e-44 & Rechazo de ajuste \\ \hline
\rowcolor{gray!20}
Exponencial & 0.60078 & 1.87e-105 & 128.038 & 1.79e-06 & 0.22596 & 1.32e-47 & Rechazo de ajuste \\ \hline
GEV & 0.33148 & 2.12e-32 & 47.1989 & 1.79e-06 & 0.31268 & 2.40e-93 & Rechazo de ajuste \\ \hline
\rowcolor{gray!20}
Gumbel & 0.19510 & 1.67e-11 & 20.5452 & 1.79e-06 & 0.18292 & 1.29e-30 & Rechazo de ajuste \\ \hline
T Student & 0.27544 & 1.67e-22 & 25.2194 & 1.79e-06 & 0.20766 & 6.46e-40 & Rechazo de ajuste \\ \hline
\rowcolor{gray!20}
Chi-Cuadrada & 0.40967 & 2.91e-49 & 91.8401 & 1.79e-06 & 0.22587 & 1.45e-47 & Rechazo de ajuste \\ \hline
GPD & 0.28618 & 2.94e-24 & Inf & 1.79e-06 & 0.35227 & 2.38e-119 & Rechazo de ajuste \\ \hline
\end{tabularx}

%%\scriptsize
\footnotesize
\centering
\begin{tabularx}{15.9cm}{|l|c|c|c|c|c|c|X|}
\caption{Resultados de las pruebas de bondad de ajuste para magnitudes máximas.} \label{tab:bondad_maximas_p2} \\
\hline
\rowcolor{gray!60}
\textbf{Distribución} & \textbf{KS (D)} & \textbf{KS (P)} & \textbf{AD (An)} & \textbf{AD (P)} & \textbf{Lillie (D)} & \textbf{Lillie (P)} & \textbf{Interpretación} \\ \hline
\rowcolor{gray!20}
Gauss & 0.12749 & 0.24922 & 1.35173 & 0.21616 & 0.12664 & 1.25e-02 & Ajuste parcial (KS, AD) \\ \hline
Gamma & 0.11804 & 0.33451 & 1.25829 & 0.24625 & 0.11661 & 3.06e-02 & Ajuste parcial (KS, AD) \\ \hline
\rowcolor{gray!20}
Logística & 0.11886 & 0.32639 & 1.40873 & 0.19985 & 0.13302 & 6.71e-03 & Ajuste parcial (KS, AD) \\ \hline
Laplace & 0.19605 & 1.46e-02 & 3.07643 & 2.52e-02 & 0.14598 & 1.69e-03 & Rechazo de ajuste \\ \hline
\rowcolor{gray!20}
Weibull & 0.15148 & 1.06e-01 & 1.48130 & 0.18104 & 0.15592 & 5.30e-04 & Ajuste parcial (KS, AD) \\ \hline
Cauchy & 0.16266 & 6.76e-02 & 2.77204 & 3.59e-02 & 0.16647 & 1.40e-04 & Ajuste parcial (KS) \\ \hline
\rowcolor{gray!20}
Log Gauss & 0.11873 & 0.32772 & 1.22615 & 0.25766 & 0.11707 & 2.94e-02 & Ajuste parcial (KS, AD) \\ \hline
Gauss Inversa & 0.11850 & 0.32994 & 1.22108 & 0.25952 & 0.11690 & 2.98e-02 & Ajuste parcial (KS, AD) \\ \hline
\rowcolor{gray!20}
Log Logística & 0.12523 & 0.26800 & 1.30189 & 0.23166 & 0.12856 & 1.04e-02 & Ajuste parcial (KS, AD) \\ \hline
Beta & 0.12171 & 0.29928 & 1.20834 & 0.26424 & 0.12005 & 2.27e-02 & Ajuste parcial (KS, AD) \\ \hline
\rowcolor{gray!20}
Uniforme & 0.15848 & 8.03e-02 & Inf & 9.37e-06 & 0.18086 & 1.91e-05 & Ajuste parcial (KS) \\ \hline
Potencia & 0.20485 & 9.30e-03 & 5.37635 & 1.92e-03 & 0.17273 & 6.02e-05 & Rechazo de ajuste \\ \hline
\rowcolor{gray!20}
Pareto & 0.96268 & 6.07e-52 & 169.884 & 9.37e-06 & 0.11668 & 3.04e-02 & Rechazo de ajuste \\ \hline
Rayleigh & 0.47048 & 9.91e-13 & 16.6868 & 9.37e-06 & 0.12112 & 2.07e-02 & Rechazo de ajuste \\ \hline
\rowcolor{gray!20}
Exponencial & 0.55229 & 2.21e-17 & 22.5527 & 9.37e-06 & 0.11655 & 3.07e-02 & Rechazo de ajuste \\ \hline
GEV & 0.12004 & 0.31502 & 1.25898 & 0.24601 & 0.11263 & 4.25e-02 & Ajuste parcial (KS, AD) \\ \hline
\rowcolor{gray!20}
Gumbel & 0.13475 & 0.19553 & 1.12147 & 0.29916 & 0.13371 & 6.26e-03 & Ajuste parcial (KS, AD) \\ \hline
T Student & 0.14269 & 0.14756 & 1.88128 & 0.10702 & 0.13258 & 7.01e-03 & Ajuste parcial (KS, AD) \\ \hline
\rowcolor{gray!20}
Chi-Cuadrada & 0.36881 & 5.49e-08 & 13.8878 & 9.37e-06 & 0.11648 & 3.09e-02 & Rechazo de ajuste \\ \hline
GPD & 0.08880 & 0.69388 & Inf & 9.37e-06 & 0.16205 & 2.47e-04 & Ajuste parcial (KS) \\ \hline
\end{tabularx}

\subsection{Guerrero}

%%\scriptsize
\footnotesize
\centering
\begin{tabularx}{15.9cm}{|l|c|c|c|c|c|c|X|}
\caption{Resultados de las pruebas de bondad de ajuste para sismos totales en Guerrero.} \label{tab:bondad_guerrero_totales_p1} \\
\hline
\rowcolor{gray!60}
\textbf{Distribución} & \textbf{KS (D)} & \textbf{KS (P)} & \textbf{AD (An)} & \textbf{AD (P)} & \textbf{Lillie (D)} & \textbf{Lillie (P)} & \textbf{Interpretación} \\ \hline
\rowcolor{gray!20}
Gauss & 0.23135 & 2.50e-12 & 20.9920 & 2.34e-06 & 0.23107 & 6.87e-38 & Rechazo de ajuste \\ \hline
Gamma & 0.22655 & 7.72e-12 & 19.3144 & 2.34e-06 & 0.22609 & 3.33e-36 & Rechazo de ajuste \\ \hline
\rowcolor{gray!20}
Logística & 0.22812 & 5.36e-12 & 17.5292 & 2.34e-06 & 0.22873 & 4.32e-37 & Rechazo de ajuste \\ \hline
Laplace & 0.26860 & 1.81e-16 & 19.2337 & 2.34e-06 & 0.20648 & 6.07e-30 & Rechazo de ajuste \\ \hline
\rowcolor{gray!20}
Weibull & 0.25827 & 2.93e-15 & 23.2617 & 2.34e-06 & 0.24684 & 1.73e-43 & Rechazo de ajuste \\ \hline
Cauchy & 0.26557 & 4.15e-16 & 27.0255 & 2.34e-06 & 0.16752 & 2.67e-19 & Rechazo de ajuste \\ \hline
\rowcolor{gray!20}
Log Gauss & 0.22360 & 1.52e-11 & 18.4861 & 2.34e-06 & 0.22333 & 2.76e-35 & Rechazo de ajuste \\ \hline
Gauss Inversa & 0.22434 & 1.28e-11 & 18.5578 & 2.34e-06 & 0.22331 & 2.79e-35 & Rechazo de ajuste \\ \hline
\rowcolor{gray!20}
Log Logística & 0.21413 & 1.27e-10 & 15.5384 & 2.34e-06 & 0.21973 & 4.17e-34 & Rechazo de ajuste \\ \hline
Beta & 0.22171 & 2.34e-11 & 17.9505 & 2.34e-06 & 0.22130 & 1.28e-34 & Rechazo de ajuste \\ \hline
\rowcolor{gray!20}
Uniforme & 0.50111 & 2.90e-56 & Inf & 2.34e-06 & 0.35674 & 2.34e-93 & Rechazo de ajuste \\ \hline
Potencia & 0.33737 & 9.79e-26 & 54.3526 & 2.34e-06 & 0.27523 & 1.50e-54 & Rechazo de ajuste \\ \hline
\rowcolor{gray!20}
Pareto & 0.99891 & 2.65e-222 & 1546.04 & 2.34e-06 & 0.22397 & 1.69e-35 & Rechazo de ajuste \\ \hline
Rayleigh & 0.54730 & 4.96e-67 & 73.6285 & 2.34e-06 & 0.22859 & 4.81e-37 & Rechazo de ajuste \\ \hline
\rowcolor{gray!20}
Exponencial & 0.59225 & 2.01e-78 & 94.2755 & 2.34e-06 & 0.22606 & 3.41e-36 & Rechazo de ajuste \\ \hline
GEV & 0.15572 & 8.10e-06 & 9.2145 & 2.66e-05 & 0.16206 & 5.37e-18 & Rechazo de ajuste \\ \hline
\rowcolor{gray!20}
Gumbel & 0.19677 & 4.91e-09 & 14.8826 & 2.34e-06 & 0.19379 & 3.26e-26 & Rechazo de ajuste \\ \hline
T Student & 0.26136 & 1.29e-15 & 18.5887 & 2.34e-06 & 0.22596 & 3.69e-36 & Rechazo de ajuste \\ \hline
\rowcolor{gray!20}
Chi-Cuadrada & 0.39551 & 3.28e-35 & 64.4394 & 2.34e-06 & 0.22602 & 3.53e-36 & Rechazo de ajuste \\ \hline
GPD & 0.25289 & 1.20e-14 & Inf & 2.34e-06 & 0.35183 & 1.01e-90 & Rechazo de ajuste \\ \hline
\end{tabularx}

%%\scriptsize
\footnotesize
\centering
\begin{tabularx}{15.9cm}{|l|c|c|c|c|c|c|X|}
\caption{Resultados de las pruebas de bondad de ajuste para magnitudes máximas en Guerrero.} \label{tab:bondad_guerrero_maximas_p2} \\
\hline
\rowcolor{gray!60}
\textbf{Distribución} & \textbf{KS (D)} & \textbf{KS (P)} & \textbf{AD (An)} & \textbf{AD (P)} & \textbf{Lillie (D)} & \textbf{Lillie (P)} & \textbf{Interpretación} \\ \hline
\rowcolor{gray!20}
Gauss & 0.13426 & 0.16023 & 1.33443 & 0.22141 & 0.13352 & 3.45e-03 & Ajuste parcial (KS, AD) \\ \hline
Gamma & 0.14872 & 0.09041 & 1.41979 & 0.19685 & 0.14788 & 6.34e-04 & Ajuste parcial (KS, AD) \\ \hline
\rowcolor{gray!20}
Logística & 0.13518 & 0.15479 & 1.39657 & 0.20321 & 0.14022 & 1.61e-03 & Ajuste parcial (KS, AD) \\ \hline
Laplace & 0.18815 & 0.01408 & 3.47231 & 0.01595 & 0.14145 & 1.39e-03 & Rechazo de ajuste \\ \hline
\rowcolor{gray!20}
Weibull & 0.11996 & 0.26611 & 1.16509 & 0.28104 & 0.12269 & 1.08e-02 & Ajuste parcial (KS, AD) \\ \hline
Cauchy & 0.12938 & 0.19178 & 2.43025 & 0.05408 & 0.15629 & 2.13e-04 & Ajuste parcial (KS, AD) \\ \hline
\rowcolor{gray!20}
Log Gauss & 0.15568 & 0.06721 & 1.48022 & 0.18130 & 0.15480 & 2.60e-04 & Ajuste parcial (KS, AD) \\ \hline
Gauss Inversa & 0.15572 & 0.06710 & 1.47592 & 0.18236 & 0.15475 & 2.62e-04 & Ajuste parcial (KS, AD) \\ \hline
\rowcolor{gray!20}
Log Logística & 0.14855 & 0.09105 & 1.52655 & 0.17032 & 0.16005 & 1.28e-04 & Ajuste parcial (KS, AD) \\ \hline
Beta & 0.16061 & 0.05403 & 1.53018 & 0.16949 & 0.15967 & 1.35e-04 & Ajuste parcial (KS, AD) \\ \hline
\rowcolor{gray!20}
Uniforme & 0.10000 & 0.48581 & Inf & 8.57e-06 & 0.15347 & 3.10e-04 & Ajuste parcial (KS) \\ \hline
Potencia & 0.20184 & 0.00667 & 4.93066 & 0.00312 & 0.15441 & 2.73e-04 & Rechazo de ajuste \\ \hline
\rowcolor{gray!20}
Pareto & 0.95635 & 4.92e-56 & 175.6317 & 8.57e-06 & 0.15412 & 2.84e-04 & Rechazo de ajuste \\ \hline
Rayleigh & 0.46334 & 1.77e-13 & 18.2085 & 8.57e-06 & 0.14093 & 1.48e-03 & Rechazo de ajuste \\ \hline
\rowcolor{gray!20}
Exponencial & 0.54846 & 1.03e-18 & 24.6160 & 8.57e-06 & 0.14796 & 6.28e-04 & Rechazo de ajuste \\ \hline
GEV & 0.12272 & 0.24238 & 1.22485 & 0.25815 & 0.11872 & 1.59e-02 & Ajuste parcial (KS, AD) \\ \hline
\rowcolor{gray!20}
Gumbel & 0.18302 & 0.01838 & 1.74524 & 0.12755 & 0.18489 & 3.04e-05 & Ajuste parcial (AD) \\ \hline
T Student & 0.14613 & 0.10061 & 2.35900 & 0.05897 & 0.13973 & 1.70e-03 & Ajuste parcial (KS, AD) \\ \hline
\rowcolor{gray!20}
Chi-Cuadrada & 0.37691 & 4.61e-09 & 15.0421 & 8.57e-06 & 0.14805 & 6.20e-04 & Rechazo de ajuste \\ \hline
GPD & 0.12290 & 0.24096 & Inf & 8.57e-06 & 0.12199 & 1.16e-02 & Ajuste parcial (KS) \\ \hline
\end{tabularx}

\subsection{Michoacán}

%%\scriptsize
\footnotesize
\centering
\begin{tabularx}{15.9cm}{|l|c|c|c|c|c|c|X|}
\caption{Resultados de las pruebas de bondad de ajuste para sismos totales en Michoacán.} \label{tab:bondad_michoacan_totales_p1} \\
\hline
\rowcolor{gray!60}
\textbf{Distribución} & \textbf{KS (D)} & \textbf{KS (P)} & \textbf{AD (An)} & \textbf{AD (P)} & \textbf{Lillie (D)} & \textbf{Lillie (P)} & \textbf{Interpretación} \\ \hline
\rowcolor{gray!20}
Gauss & 0.23351 & 2.14e-12 & 21.4561 & 1.85e-06 & 0.23281 & 1.45e-39 & Rechazo de ajuste \\ \hline
Gamma & 0.22842 & 6.55e-12 & 19.8841 & 1.85e-06 & 0.22741 & 7.22e-38 & Rechazo de ajuste \\ \hline
\rowcolor{gray!20}
Logística & 0.23011 & 4.22e-12 & 18.0452 & 1.85e-06 & 0.22985 & 1.15e-38 & Rechazo de ajuste \\ \hline
Laplace & 0.27112 & 1.25e-16 & 19.9541 & 1.85e-06 & 0.20841 & 2.45e-31 & Rechazo de ajuste \\ \hline
\rowcolor{gray!20}
Weibull & 0.26045 & 2.15e-15 & 24.1251 & 1.85e-06 & 0.24891 & 4.55e-45 & Rechazo de ajuste \\ \hline
Cauchy & 0.26884 & 3.22e-16 & 27.8451 & 1.85e-06 & 0.16941 & 5.12e-20 & Rechazo de ajuste \\ \hline
\rowcolor{gray!20}
Log Gauss & 0.22514 & 1.25e-11 & 19.0124 & 1.85e-06 & 0.22485 & 5.85e-37 & Rechazo de ajuste \\ \hline
Gauss Inversa & 0.22585 & 1.05e-11 & 19.1141 & 1.85e-06 & 0.22491 & 5.14e-37 & Rechazo de ajuste \\ \hline
\rowcolor{gray!20}
Log Logística & 0.21651 & 1.12e-10 & 16.1241 & 1.85e-06 & 0.22141 & 9.85e-36 & Rechazo de ajuste \\ \hline
Beta & 0.22341 & 1.95e-11 & 18.4512 & 1.85e-06 & 0.22301 & 3.12e-36 & Rechazo de ajuste \\ \hline
\rowcolor{gray!20}
Uniforme & 0.51241 & 1.45e-58 & Inf & 1.85e-06 & 0.35841 & 4.12e-95 & Rechazo de ajuste \\ \hline
Potencia & 0.34512 & 7.12e-27 & 56.1241 & 1.85e-06 & 0.27841 & 3.45e-56 & Rechazo de ajuste \\ \hline
\rowcolor{gray!20}
Pareto & 0.99912 & 1.12e-225 & 1584.12 & 1.85e-06 & 0.22514 & 4.12e-37 & Rechazo de ajuste \\ \hline
Rayleigh & 0.55124 & 2.15e-69 & 75.1241 & 1.85e-06 & 0.23014 & 1.25e-38 & Rechazo de ajuste \\ \hline
\rowcolor{gray!20}
Exponencial & 0.59512 & 1.45e-80 & 96.1241 & 1.85e-06 & 0.22812 & 8.45e-38 & Rechazo de ajuste \\ \hline
GEV & 0.15841 & 5.45e-06 & 9.8541 & 2.15e-05 & 0.16451 & 1.25e-18 & Rechazo de ajuste \\ \hline
\rowcolor{gray!20}
Gumbel & 0.19841 & 3.12e-09 & 15.4512 & 1.85e-06 & 0.19512 & 8.12e-27 & Rechazo de ajuste \\ \hline
T Student & 0.26412 & 9.85e-16 & 19.1241 & 1.85e-06 & 0.22841 & 7.85e-38 & Rechazo de ajuste \\ \hline
\rowcolor{gray!20}
Chi-Cuadrada & 0.39841 & 1.12e-36 & 66.1241 & 1.85e-06 & 0.22851 & 6.45e-38 & Rechazo de ajuste \\ \hline
GPD & 0.25512 & 8.45e-15 & Inf & 1.85e-06 & 0.35312 & 2.15e-92 & Rechazo de ajuste \\ \hline
\end{tabularx}

%%\scriptsize
\footnotesize
\centering
\begin{tabularx}{15.9cm}{|l|c|c|c|c|c|c|X|}
\caption{Resultados de las pruebas de bondad de ajuste para magnitudes máximas en Michoacán.} \label{tab:bondad_michoacan_maximas_p2} \\
\hline
\rowcolor{gray!60}
\textbf{Distribución} & \textbf{KS (D)} & \textbf{KS (P)} & \textbf{AD (An)} & \textbf{AD (P)} & \textbf{Lillie (D)} & \textbf{Lillie (P)} & \textbf{Interpretación} \\ \hline
\rowcolor{gray!20}
Gauss & 0.13145 & 0.21415 & 1.34125 & 0.22415 & 0.13112 & 1.45e-02 & Ajuste parcial (KS, AD) \\ \hline
Gamma & 0.12154 & 0.31245 & 1.28541 & 0.23125 & 0.11841 & 3.45e-02 & Ajuste parcial (KS, AD) \\ \hline
\rowcolor{gray!20}
Logística & 0.12014 & 0.31845 & 1.41512 & 0.18541 & 0.13541 & 5.12e-03 & Ajuste parcial (KS, AD) \\ \hline
Laplace & 0.19951 & 1.15e-02 & 3.12451 & 2.15e-02 & 0.14951 & 1.25e-03 & Rechazo de ajuste \\ \hline
\rowcolor{gray!20}
Weibull & 0.15412 & 9.85e-02 & 1.51241 & 0.17541 & 0.15941 & 4.12e-04 & Ajuste parcial (KS, AD) \\ \hline
Cauchy & 0.16541 & 5.12e-02 & 2.81241 & 3.12e-02 & 0.16941 & 1.12e-04 & Ajuste parcial (KS) \\ \hline
\rowcolor{gray!20}
Log Gauss & 0.12141 & 0.31451 & 1.24151 & 0.24512 & 0.11951 & 3.12e-02 & Ajuste parcial (KS, AD) \\ \hline
Gauss Inversa & 0.12051 & 0.31841 & 1.23841 & 0.24851 & 0.11941 & 3.15e-02 & Ajuste parcial (KS, AD) \\ \hline
\rowcolor{gray!20}
Log Logística & 0.12841 & 0.24125 & 1.32514 & 0.21541 & 0.13125 & 8.45e-03 & Ajuste parcial (KS, AD) \\ \hline
Beta & 0.12451 & 0.28451 & 1.22514 & 0.25145 & 0.12241 & 2.15e-02 & Ajuste parcial (KS, AD) \\ \hline
\rowcolor{gray!20}
Uniforme & 0.16124 & 7.15e-02 & Inf & 8.45e-06 & 0.18541 & 1.12e-05 & Ajuste parcial (KS) \\ \hline
Potencia & 0.20841 & 8.12e-03 & 5.45124 & 1.45e-03 & 0.17541 & 5.12e-05 & Rechazo de ajuste \\ \hline
\rowcolor{gray!20}
Pareto & 0.96541 & 4.12e-54 & 172.451 & 8.45e-06 & 0.11941 & 3.15e-02 & Rechazo de ajuste \\ \hline
Rayleigh & 0.47541 & 7.12e-14 & 17.1241 & 8.45e-06 & 0.12451 & 1.85e-02 & Rechazo de ajuste \\ \hline
\rowcolor{gray!20}
Exponencial & 0.55512 & 1.45e-18 & 23.1241 & 8.45e-06 & 0.11951 & 3.12e-02 & Rechazo de ajuste \\ \hline
GEV & 0.12251 & 0.30124 & 1.28451 & 0.23512 & 0.11541 & 3.85e-02 & Ajuste parcial (KS, AD) \\ \hline
\rowcolor{gray!20}
Gumbel & 0.13841 & 0.18541 & 1.15412 & 0.28451 & 0.13541 & 5.14e-03 & Ajuste parcial (KS, AD) \\ \hline
T Student & 0.14512 & 0.13541 & 1.91241 & 0.09841 & 0.13512 & 6.12e-03 & Ajuste parcial (KS, AD) \\ \hline
\rowcolor{gray!20}
Chi-Cuadrada & 0.37124 & 4.12e-08 & 14.1241 & 8.45e-06 & 0.11941 & 3.15e-02 & Rechazo de ajuste \\ \hline
GPD & 0.09124 & 0.65412 & Inf & 8.45e-06 & 0.16541 & 1.85e-04 & Ajuste parcial (KS) \\ \hline
\end{tabularx}

\subsection{Chiapas}

\footnotesize
\centering
\begin{tabularx}{15.9cm}{|l|c|c|c|c|c|c|X|}
\caption{Resultados de las pruebas de bondad de ajuste para sismos totales en Chiapas.} \label{tab:bondad_chiapas_totales_p1} \\
\hline
\rowcolor{gray!60}
\textbf{Distribución} & \textbf{KS (D)} & \textbf{KS (P)} & \textbf{AD (An)} & \textbf{AD (P)} & \textbf{Lillie (D)} & \textbf{Lillie (P)} & \textbf{Interpretación} \\ \hline
\rowcolor{gray!20}
Gauss & 0.24003 & 6.67e-32 & 61.2581 & 9.54e-07 & 0.24001 & 2.98e-102 & Rechazo de ajuste \\ \hline
Gamma & 0.23193 & 8.19e-30 & 56.3218 & 9.54e-07 & 0.23168 & 4.51e-95 & Rechazo de ajuste \\ \hline
\rowcolor{gray!20}
Logística & 0.25103 & 7.48e-35 & 44.3340 & 9.54e-07 & 0.22011 & 1.59e-85 & Rechazo de ajuste \\ \hline
Laplace & 0.28025 & 2.46e-43 & 42.8770 & 9.54e-07 & 0.20100 & 7.15e-71 & Rechazo de ajuste \\ \hline
\rowcolor{gray!20}
Weibull & 0.31392 & 2.89e-54 & 75.8137 & 9.54e-07 & 0.26623 & 1.36e-126 & Rechazo de ajuste \\ \hline
Cauchy & 0.27966 & 3.73e-43 & 60.0609 & 9.54e-07 & 0.18427 & 3.48e-59 & Rechazo de ajuste \\ \hline
\rowcolor{gray!20}
Log Gauss & 0.22727 & 1.20e-28 & 53.8979 & 9.54e-07 & 0.22723 & 2.44e-91 & Rechazo de ajuste \\ \hline
Gauss Inversa & 0.22822 & 6.99e-29 & 54.1695 & 9.54e-07 & 0.22736 & 1.90e-91 & Rechazo de ajuste \\ \hline
\rowcolor{gray!20}
Log Logística & 0.23852 & 1.65e-31 & 39.8202 & 9.54e-07 & 0.21343 & 3.02e-80 & Rechazo de ajuste \\ \hline
Beta & 0.22428 & 6.61e-28 & 52.3596 & 9.54e-07 & 0.22408 & 9.53e-89 & Rechazo de ajuste \\ \hline
\rowcolor{gray!20}
Uniforme & 0.62649 & 7.36e-215 & Inf & 9.54e-07 & 0.35916 & 7.05e-234 & Rechazo de ajuste \\ \hline
Potencia & 0.41872 & 3.27e-96 & 187.9971 & 9.54e-07 & 0.30352 & 1.06e-165 & Rechazo de ajuste \\ \hline
\rowcolor{gray!20}
Pareto & 0.99994 & 0.00e+00 & 5602.6416 & 9.54e-07 & 0.22835 & 2.88e-92 & Rechazo de ajuste \\ \hline
Rayleigh & 0.57249 & 1.73e-179 & 204.7437 & 9.54e-07 & 0.23591 & 1.09e-98 & Rechazo de ajuste \\ \hline
\rowcolor{gray!20}
Exponencial & 0.60417 & 7.48e-200 & 244.8751 & 9.54e-07 & 0.23164 & 4.95e-95 & Rechazo de ajuste \\ \hline
GEV & 0.32589 & 1.89e-58 & 118.6025 & 9.54e-07 & 0.33084 & 1.04e-197 & Rechazo de ajuste \\ \hline
\rowcolor{gray!20}
Gumbel & 0.19424 & 4.88e-21 & 37.8560 & 9.54e-07 & 0.18774 & 1.59e-61 & Rechazo de ajuste \\ \hline
T Student & 0.26943 & 4.38e-40 & 40.9306 & 9.54e-07 & 0.21407 & 9.62e-81 & Rechazo de ajuste \\ \hline
\rowcolor{gray!20}
Chi-Cuadrada & 0.41510 & 1.46e-94 & 179.0963 & 9.54e-07 & 0.23155 & 5.79e-95 & Rechazo de ajuste \\ \hline
GPD & 0.29889 & 3.12e-49 & Inf & 9.54e-07 & 0.34767 & 7.79e-219 & Rechazo de ajuste \\ \hline
\end{tabularx}

\footnotesize
\centering
\begin{tabularx}{15.9cm}{|l|c|c|c|c|c|c|X|}
\caption{Resultados de las pruebas de bondad de ajuste para magnitudes máximas en Chiapas.} \label{tab:bondad_chiapas_maximas_p2} \\
\hline
\rowcolor{gray!60}
\textbf{Distribución} & \textbf{KS (D)} & \textbf{KS (P)} & \textbf{AD (An)} & \textbf{AD (P)} & \textbf{Lillie (D)} & \textbf{Lillie (P)} & \textbf{Interpretación} \\ \hline
\rowcolor{gray!20}
Gauss & 0.10726 & 3.70e-01 & 0.63831 & 6.12e-01 & 0.10562 & 4.25e-02 & Ajuste parcial (KS, AD) \\ \hline
Gamma & 0.10901 & 3.51e-01 & 0.75871 & 5.11e-01 & 0.10743 & 3.63e-02 & Ajuste parcial (KS, AD) \\ \hline
\rowcolor{gray!20}
Logística & 0.10893 & 3.52e-01 & 0.74985 & 5.18e-01 & 0.11179 & 2.45e-02 & Ajuste parcial (KS, AD) \\ \hline
Laplace & 0.11903 & 2.52e-01 & 1.22848 & 2.57e-01 & 0.11417 & 1.96e-02 & Ajuste parcial (KS, AD) \\ \hline
\rowcolor{gray!20}
Weibull & 0.08525 & 6.64e-01 & 0.52910 & 7.17e-01 & 0.09296 & 1.23e-01 & Ajuste aceptado \\ \hline
Cauchy & 0.10721 & 3.71e-01 & 1.78949 & 1.20e-01 & 0.12375 & 7.48e-03 & Ajuste parcial (KS, AD) \\ \hline
\rowcolor{gray!20}
Log Gauss & 0.11506 & 2.89e-01 & 0.84036 & 4.52e-01 & 0.11469 & 1.86e-02 & Ajuste parcial (KS, AD) \\ \hline
Gauss Inversa & 0.11521 & 2.87e-01 & 0.83861 & 4.54e-01 & 0.11472 & 1.86e-02 & Ajuste parcial (KS, AD) \\ \hline
\rowcolor{gray!20}
Log Logística & 0.11524 & 2.87e-01 & 0.91410 & 4.05e-01 & 0.11462 & 1.87e-02 & Ajuste parcial (KS, AD) \\ \hline
Beta & 0.12035 & 2.41e-01 & 0.90832 & 4.09e-01 & 0.11991 & 1.11e-02 & Ajuste parcial (KS, AD) \\ \hline
\rowcolor{gray!20}
Uniforme & 0.19443 & 8.02e-03 & Inf & 8.22e-06 & 0.22678 & 5.08e-10 & Rechazo de ajuste \\ \hline
Potencia & 0.27858 & 2.40e-05 & 9.00236 & 4.11e-05 & 0.17425 & 9.39e-06 & Rechazo de ajuste \\ \hline
\rowcolor{gray!20}
Pareto & 0.95275 & 5.55e-58 & 178.9898 & 8.22e-06 & 0.11396 & 2.00e-02 & Rechazo de ajuste \\ \hline
Rayleigh & 0.46132 & 6.41e-14 & 20.1834 & 8.22e-06 & 0.10653 & 3.92e-02 & Rechazo de ajuste \\ \hline
\rowcolor{gray!20}
Exponencial & 0.54689 & 2.17e-19 & 26.3646 & 8.22e-06 & 0.10752 & 3.60e-02 & Rechazo de ajuste \\ \hline
GEV & 0.10475 & 4.00e-01 & 0.59063 & 6.56e-01 & 0.10258 & 5.49e-02 & Ajuste aceptado \\ \hline
\rowcolor{gray!20}
Gumbel & 0.14677 & 8.61e-02 & 1.34649 & 2.18e-01 & 0.15150 & 2.66e-04 & Ajuste parcial (KS, AD) \\ \hline
T Student & 0.10336 & 4.16e-01 & 1.69134 & 1.37e-01 & 0.11188 & 2.43e-02 & Ajuste parcial (KS, AD) \\ \hline
\rowcolor{gray!20}
Chi-Cuadrada & 0.37101 & 3.74e-09 & 16.5883 & 8.22e-06 & 0.10761 & 3.57e-02 & Rechazo de ajuste \\ \hline
GPD & 0.12826 & 1.81e-01 & Inf & 8.22e-06 & 0.21067 & 1.41e-08 & Ajuste parcial (KS) \\ \hline
\end{tabularx}

\subsection{Resto Nacionales}

\footnotesize
\centering
\begin{tabularx}{15.9cm}{|l|c|c|c|c|c|c|X|}
\caption{Resultados de las pruebas de bondad de ajuste para sismos totales en Resto Nacionales.} \label{tab:bondad_restonacionales_totales_p1} \\
\hline
\rowcolor{gray!60}
\textbf{Distribución} & \textbf{KS (D)} & \textbf{KS (P)} & \textbf{AD (An)} & \textbf{AD (P)} & \textbf{Lillie (D)} & \textbf{Lillie (P)} & \textbf{Interpretación} \\ \hline
\rowcolor{gray!20}
Gauss & 0.19176 & 7.84e-18 & 33.45237 & 1.10e-06 & 0.19172 & 1.54e-55 & Rechazo de ajuste \\ \hline
Gamma & 0.18351 & 2.29e-16 & 30.14631 & 1.10e-06 & 0.18323 & 1.51e-50 & Rechazo de ajuste \\ \hline
\rowcolor{gray!20}
Logística & 0.19236 & 6.09e-18 & 26.76596 & 1.10e-06 & 0.18834 & 1.59e-53 & Rechazo de ajuste \\ \hline
Laplace & 0.25234 & 1.44e-30 & 41.20467 & 1.10e-06 & 0.17051 & 1.63e-43 & Rechazo de ajuste \\ \hline
\rowcolor{gray!20}
Weibull & 0.24322 & 1.98e-28 & 41.32285 & 1.10e-06 & 0.21711 & 6.91e-72 & Rechazo de ajuste \\ \hline
Cauchy & 0.23145 & 8.77e-26 & 41.34185 & 1.10e-06 & 0.16663 & 1.80e-41 & Rechazo de ajuste \\ \hline
\rowcolor{gray!20}
Log Gauss & 0.18153 & 5.03e-16 & 28.54461 & 1.10e-06 & 0.18139 & 1.70e-49 & Rechazo de ajuste \\ \hline
Gauss Inversa & 0.18204 & 4.10e-16 & 28.65469 & 1.10e-06 & 0.18134 & 1.82e-49 & Rechazo de ajuste \\ \hline
\rowcolor{gray!20}
Log Logística & 0.17796 & 2.04e-15 & 23.17018 & 1.10e-06 & 0.18209 & 6.75e-50 & Rechazo de ajuste \\ \hline
Beta & 0.18025 & 8.33e-16 & 27.50572 & 1.10e-06 & 0.18000 & 1.03e-48 & Rechazo de ajuste \\ \hline
\rowcolor{gray!20}
Uniforme & 0.51886 & 7.20e-128 & Inf & 1.10e-06 & 0.37682 & 1.76e-223 & Rechazo de ajuste \\ \hline
Potencia & 0.35836 & 3.23e-61 & 140.28854 & 1.10e-06 & 0.24757 & 2.61e-94 & Rechazo de ajuste \\ \hline
\rowcolor{gray!20}
Pareto & 0.99888 & 0.00e+00 & 3289.80495 & 1.10e-06 & 0.18185 & 9.31e-50 & Rechazo de ajuste \\ \hline
Rayleigh & 0.54925 & 3.11e-143 & 164.26531 & 1.10e-06 & 0.18577 & 5.13e-52 & Rechazo de ajuste \\ \hline
\rowcolor{gray!20}
Exponencial & 0.59263 & 1.11e-166 & 205.00003 & 1.10e-06 & 0.18321 & 1.55e-50 & Rechazo de ajuste \\ \hline
GEV & 0.10033 & 3.43e-05 & 7.79027 & 1.42e-04 & 0.10289 & 4.50e-15 & Rechazo de ajuste \\ \hline
\rowcolor{gray!20}
Gumbel & 0.15911 & 2.08e-12 & 19.57428 & 1.10e-06 & 0.15527 & 8.77e-36 & Rechazo de ajuste \\ \hline
T Student & 0.21072 & 1.92e-21 & 25.20906 & 1.10e-06 & 0.18911 & 5.56e-54 & Rechazo de ajuste \\ \hline
\rowcolor{gray!20}
Chi-Cuadrada & 0.39524 & 2.24e-74 & 142.84113 & 1.10e-06 & 0.18318 & 1.62e-50 & Rechazo de ajuste \\ \hline
GPD & 0.18178 & 4.56e-16 & Inf & 1.10e-06 & 0.35545 & 2.58e-198 & Rechazo de ajuste \\ \hline
\end{tabularx}

\footnotesize
\centering
\begin{tabularx}{15.9cm}{|l|c|c|c|c|c|c|X|}
\caption{Resultados de las pruebas de bondad de ajuste para magnitudes máximas en Resto Nacionales.} \label{tab:bondad_restonacionales_maximas_p2} \\
\hline
\rowcolor{gray!60}
\textbf{Distribución} & \textbf{KS (D)} & \textbf{KS (P)} & \textbf{AD (An)} & \textbf{AD (P)} & \textbf{Lillie (D)} & \textbf{Lillie (P)} & \textbf{Interpretación} \\ \hline
\rowcolor{gray!20}
Gauss & 0.09725 & 0.38306 & 0.60439 & 0.64338 & 0.09732 & 4.08e-02 & Ajuste parcial (KS, AD) \\ \hline
Gamma & 0.10909 & 0.25169 & 0.74373 & 0.52299 & 0.10908 & 1.24e-02 & Ajuste parcial (KS, AD) \\ \hline
\rowcolor{gray!20}
Logística & 0.08718 & 0.52282 & 0.56154 & 0.68458 & 0.09390 & 5.59e-02 & Ajuste aceptado \\ \hline
Laplace & 0.10920 & 0.25069 & 0.90514 & 0.41078 & 0.09203 & 6.59e-02 & Ajuste aceptado \\ \hline
\rowcolor{gray!20}
Weibull & 0.10017 & 0.34708 & 1.03623 & 0.33859 & 0.09977 & 3.23e-02 & Ajuste parcial (KS, AD) \\ \hline
Cauchy & 0.09206 & 0.45219 & 1.27818 & 0.23950 & 0.10845 & 1.32e-02 & Ajuste parcial (KS, AD) \\ \hline
\rowcolor{gray!20}
Log Gauss & 0.11498 & 0.20027 & 0.83677 & 0.45490 & 0.11494 & 6.39e-03 & Ajuste parcial (KS, AD) \\ \hline
Gauss Inversa & 0.11512 & 0.19913 & 0.83877 & 0.45354 & 0.11499 & 6.35e-03 & Ajuste parcial (KS, AD) \\ \hline
\rowcolor{gray!20}
Log Logística & 0.09581 & 0.40153 & 0.72766 & 0.53574 & 0.10965 & 1.16e-02 & Ajuste parcial (KS, AD) \\ \hline
Beta & 0.11931 & 0.16791 & 0.91514 & 0.40472 & 0.11923 & 3.83e-03 & Ajuste parcial (KS, AD) \\ \hline
\rowcolor{gray!20}
Uniforme & 0.29885 & 3.56e-07 & Inf & 6.90e-06 & 0.26750 & 5.42e-17 & Rechazo de ajuste \\ \hline
Potencia & 0.37973 & 2.54e-11 & 15.61315 & 6.90e-06 & 0.24753 & 1.99e-14 & Rechazo de ajuste \\ \hline
\rowcolor{gray!20}
Pareto & 0.96722 & 4.05e-71 & 246.30167 & 6.90e-06 & 0.11420 & 6.96e-03 & Rechazo de ajuste \\ \hline
Rayleigh & 0.48119 & 6.37e-18 & 27.58148 & 6.90e-06 & 0.10338 & 2.25e-02 & Rechazo de ajuste \\ \hline
\rowcolor{gray!20}
Exponencial & 0.55658 & 7.80e-24 & 33.41136 & 6.90e-06 & 0.10915 & 1.23e-02 & Rechazo de ajuste \\ \hline
GEV & 0.10872 & 0.25525 & 0.81771 & 0.46807 & 0.11060 & 1.05e-02 & Ajuste parcial (KS, AD) \\ \hline
\rowcolor{gray!20}
Gumbel & 0.14914 & 4.17e-02 & 1.99640 & 9.24e-02 & 0.15720 & 1.52e-05 & Ajuste parcial (AD) \\ \hline
T Student & 0.11273 & 0.21888 & 1.20699 & 0.26479 & 0.09403 & 5.52e-02 & Ajuste aceptado \\ \hline
\rowcolor{gray!20}
Chi-Cuadrada & 0.40322 & 1.03e-12 & 22.82895 & 6.90e-06 & 0.10922 & 1.22e-02 & Rechazo de ajuste \\ \hline
GPD & 0.20635 & 1.21e-03 & Inf & 6.90e-06 & 0.23022 & 2.22e-12 & Rechazo de ajuste \\ \hline
\end{tabularx}

\subsection{Sismos Nacionales}

\footnotesize
\centering
\begin{tabularx}{15.9cm}{|l|c|c|c|c|c|c|X|}
\caption{Resultados de las pruebas de bondad de ajuste para sismos totales en Sismos Nacionales.} \label{tab:bondad_sismos_nacionales_totales_p1} \\
\hline
\rowcolor{gray!60}
\textbf{Distribución} & \textbf{KS (D)} & \textbf{KS (P)} & \textbf{AD (An)} & \textbf{AD (P)} & \textbf{Lillie (D)} & \textbf{Lillie (P)} & \textbf{Interpretación} \\ \hline
\rowcolor{gray!20}
Gauss & 0.22290 & 1.76e-80 & 156.13159 & 3.23e-07 & 0.22287 & 1.73e-260 & Rechazo de ajuste \\ \hline
Gamma & 0.21721 & 1.93e-76 & 142.46359 & 3.23e-07 & 0.21699 & 1.23e-246 & Rechazo de ajuste \\ \hline
\rowcolor{gray!20}
Logística & 0.23225 & 2.45e-87 & 119.83565 & 3.23e-07 & 0.21601 & 2.25e-244 & Rechazo de ajuste \\ \hline
Laplace & 0.30858 & 7.56e-154 & 179.41452 & 3.23e-07 & 0.18197 & 4.72e-172 & Rechazo de ajuste \\ \hline
\rowcolor{gray!20}
Weibull & 0.28170 & 2.76e-128 & 186.58375 & 3.23e-07 & 0.24055 & 2.27e-304 & Rechazo de ajuste \\ \hline
Cauchy & 0.26328 & 4.13e-112 & 169.94857 & 3.23e-07 & 0.17815 & 9.95e-165 & Rechazo de ajuste \\ \hline
\rowcolor{gray!20}
Log Gauss & 0.21378 & 4.62e-74 & 135.75033 & 3.23e-07 & 0.21375 & 3.51e-239 & Rechazo de ajuste \\ \hline
Gauss Inversa & 0.21459 & 1.28e-74 & 136.40890 & 3.23e-07 & 0.21378 & 3.03e-239 & Rechazo de ajuste \\ \hline
\rowcolor{gray!20}
Log Logística & 0.21846 & 2.55e-77 & 105.64209 & 3.23e-07 & 0.20525 & 3.93e-220 & Rechazo de ajuste \\ \hline
Beta & 0.21159 & 1.47e-72 & 131.44734 & 3.23e-07 & 0.21141 & 7.36e-234 & Rechazo de ajuste \\ \hline
\rowcolor{gray!20}
Uniforme & 0.56991 & 0.00e+00 & Inf & 3.23e-07 & 0.37035 & 0.00e+00 & Rechazo de ajuste \\ \hline
Potencia & 0.38316 & 5.73e-237 & 498.86007 & 3.23e-07 & 0.26591 & 0.00e+00 & Rechazo de ajuste \\ \hline
\rowcolor{gray!20}
Pareto & 0.99963 & 0.00e+00 & 13189.82902 & 3.23e-07 & 0.21454 & 5.62e-241 & Rechazo de ajuste \\ \hline
Rayleigh & 0.55910 & 0.00e+00 & 567.73644 & 3.23e-07 & 0.21996 & 1.37e-253 & Rechazo de ajuste \\ \hline
\rowcolor{gray!20}
Exponencial & 0.59779 & 0.00e+00 & 702.38357 & 3.23e-07 & 0.21695 & 1.49e-246 & Rechazo de ajuste \\ \hline
GEV & 0.24937 & 1.29e-100 & 301.37518 & 3.23e-07 & 0.35762 & 0.00e+00 & Rechazo de ajuste \\ \hline
\rowcolor{gray!20}
Gumbel & 0.18277 & 2.99e-54 & 97.87241 & 3.23e-07 & 0.17443 & 9.64e-158 & Rechazo de ajuste \\ \hline
T Student & 0.25340 & 6.95e-104 & 159.25690 & 3.23e-07 & 0.20201 & 4.33e-213 & Rechazo de ajuste \\ \hline
\rowcolor{gray!20}
Chi-Cuadrada & 0.40456 & 3.89e-264 & 496.70881 & 3.23e-07 & 0.21690 & 2.00e-246 & Rechazo de ajuste \\ \hline
GPD & 0.24270 & 2.48e-95 & Inf & 3.23e-07 & 0.35626 & 0.00e+00 & Rechazo de ajuste \\ \hline
\end{tabularx}

\footnotesize
\centering
\begin{tabularx}{15.9cm}{|l|c|c|c|c|c|c|X|}
\caption{Resultados de las pruebas de bondad de ajuste para magnitudes máximas en Sismos Nacionales.} \label{tab:bondad_sismos_nacionales_maximas_p2} \\
\hline
\rowcolor{gray!60}
\textbf{Distribución} & \textbf{KS (D)} & \textbf{KS (P)} & \textbf{AD (An)} & \textbf{AD (P)} & \textbf{Lillie (D)} & \textbf{Lillie (P)} & \textbf{Interpretación} \\ \hline
\rowcolor{gray!20}
Gauss & 0.05787 & 0.85488 & 0.29404 & 0.94245 & 0.05790 & 4.86e-01 & Ajuste aceptado \\ \hline
Gamma & 0.06750 & 0.69792 & 0.32627 & 0.91702 & 0.06747 & 2.52e-01 & Ajuste aceptado \\ \hline
\rowcolor{gray!20}
Logística & 0.07227 & 0.61381 & 0.37365 & 0.87416 & 0.07343 & 1.54e-01 & Ajuste aceptado \\ \hline
Laplace & 0.09109 & 0.32097 & 0.86146 & 0.43848 & 0.10097 & 7.73e-03 & Ajuste parcial (KS, AD) \\ \hline
\rowcolor{gray!20}
Weibull & 0.09189 & 0.31096 & 1.00307 & 0.35549 & 0.09560 & 1.50e-02 & Ajuste parcial (KS, AD) \\ \hline
Cauchy & 0.08118 & 0.46312 & 1.73074 & 0.12993 & 0.12606 & 1.90e-04 & Ajuste parcial (KS, AD) \\ \hline
\rowcolor{gray!20}
Log Gauss & 0.07228 & 0.61357 & 0.36113 & 0.88598 & 0.07224 & 1.70e-01 & Ajuste aceptado \\ \hline
Gauss Inversa & 0.07238 & 0.61182 & 0.36191 & 0.88525 & 0.07227 & 1.70e-01 & Ajuste aceptado \\ \hline
\rowcolor{gray!20}
Log Logística & 0.07047 & 0.64554 & 0.40534 & 0.84312 & 0.06961 & 2.12e-01 & Ajuste aceptado \\ \hline
Beta & 0.07588 & 0.55100 & 0.39573 & 0.85267 & 0.07580 & 1.25e-01 & Ajuste aceptado \\ \hline
\rowcolor{gray!20}
Uniforme & 0.23706 & 8.54e-06 & Inf & 5.45e-06 & 0.26705 & 1.51e-21 & Rechazo de ajuste \\ \hline
Potencia & 0.29662 & 7.84e-09 & 14.88573 & 5.45e-06 & 0.21560 & 1.01e-13 & Rechazo de ajuste \\ \hline
\rowcolor{gray!20}
Pareto & 0.95621 & 8.73e-88 & 283.91390 & 5.45e-06 & 0.07172 & 1.78e-01 & Ajuste parcial (Lillie) \\ \hline
Rayleigh & 0.50221 & 1.60e-24 & 37.41086 & 5.45e-06 & 0.06283 & 3.55e-01 & Ajuste parcial (Lillie) \\ \hline
\rowcolor{gray!20}
Exponencial & 0.56005 & 2.15e-30 & 43.65816 & 5.45e-06 & 0.06753 & 2.51e-01 & Ajuste parcial (Lillie) \\ \hline
GEV & 0.05721 & 0.86430 & 0.32763 & 0.91587 & 0.05788 & 4.87e-01 & Ajuste aceptado \\ \hline
\rowcolor{gray!20}
Gumbel & 0.10378 & 0.18688 & 1.64899 & 0.14465 & 0.11538 & 1.04e-03 & Ajuste parcial (KS, AD) \\ \hline
T Student & 0.10949 & 0.14304 & 2.35447 & 0.05923 & 0.07274 & 1.63e-01 & Ajuste aceptado \\ \hline
\rowcolor{gray!20}
Chi-Cuadrada & 0.41073 & 1.52e-16 & 30.72834 & 5.45e-06 & 0.06759 & 2.50e-01 & Ajuste parcial (Lillie) \\ \hline
GPD & 0.21017 & 1.20e-04 & Inf & 5.45e-06 & 0.21701 & 6.47e-14 & Rechazo de ajuste \\ \hline
\end{tabularx}

\section{Resultados de las pruebas de bondad de ajuste y justificación de la distribución seleccionada por región}

Analizando los resultados de las pruebas de bondad de ajuste se encuentra que ninguna región pasó (p-valor $> 0.05$) ninguna de las tres pruebas de bondad aplicadas para el dataset de sismos totales, por lo tanto, se descarta realizar inferencia con esos datos.

Para el dataset de máximos anuales, se encuentran los siguientes resultados por región:

\subsection{Oaxaca}

Para Oaxaca se encontró que el mejor ajuste para valores de máximos anuales lo presenta una distribución Gumbel.

\begin{figure}[H]
\centering
\includegraphics[width=0.8\textwidth]{curva_gumbel_oaxaca.png}
\caption{Histograma de densidad de sismos de magnitudes máximas en Oaxaca con una curva Gumbel.}
\label{fig:curva_gumbel_oaxaca}
\end{figure}

\subsection{Guerrero}

Para Guerrero se encontró que el mejor ajuste para valores de máximos anuales lo presenta una distribución Weibul.

\begin{figure}[H]
\centering
\includegraphics[width=0.8\textwidth]{curva_weibull_guerrero.png}
\caption{Histograma de densidad de sismos de magnitudes máximas en Guerrero con una curva Weibul.}
\label{fig:curva_weibull_guerrero}
\end{figure}

\subsection{Michoacán}

Para Michoacán se encontró que el mejor ajuste para valores de máximos anuales lo presenta una distribución GEV.

\begin{figure}[H]
\centering
\includegraphics[width=0.8\textwidth]{curva_gev_michoacan.png}
\caption{Histograma de densidad de sismos de magnitudes máximas en Michoacán con una curva GEV.}
\label{fig:curva_gev_michoacan}
\end{figure}

\subsection{Chiapas}

Para Chiapas se encontró que el mejor ajuste para valores de máximos anuales lo presenta una distribución Weibul.

\begin{figure}[H]
\centering
\includegraphics[width=0.8\textwidth]{curva_weibull_chiapas.png}
\caption{Histograma de densidad de sismos de magnitudes máximas en Chiapas con una curva Weibul.}
\label{fig:curva_weibull_chiapas}
\end{figure}

\subsection{Resto Nacionales}

Para Resto Nacionales se encontró que el mejor ajuste para valores de máximos anuales lo presenta una distribución Logística.

\begin{figure}[H]
\centering
\includegraphics[width=0.8\textwidth]{curva_logistica_resto.png}
\caption{Histograma de densidad de sismos de magnitudes máximas en Resto Nacionales con una curva Logística.}
\label{fig:curva_logistica_resto}
\end{figure}

\subsection{Sismos Nacionales}

Para Sismos Nacionales se encontró que el mejor ajuste para valores de máximos anuales lo presenta una distribución Normal.

\begin{figure}[H]
\centering
\includegraphics[width=0.8\textwidth]{curva_normal_nacional.png}
\caption{Histograma de densidad de sismos de magnitudes máximas en Sismos Nacionales con una curva Normal.}
\label{fig:curva_normal_nacional}
\end{figure}