\chapter{Análisis de resultados \label{cap:AnalisisDeResultados}}}
\noindent
El proceso de analizar y estudiar los datos sísmicos recopilados del SSN y luego aplicar la metodología descrita en el capítulo anterior nos lleva a la obtención de distintos resultados con los cuales se busca responder a las preguntas de investigación planteadas. Este capítulo está dividido en diferentes secciones con el propósito de esclarecer el uso de las herramientas implementadas, los conceptos analíticos y su interpretación.
\section{Herramientas utilizadas}
\noindent
El archivo recopilado desde el sitio del SSN es un archivo en formato CSV bruto el cual contiene la información estadística de todos los sismos del país desde el 01 de enero de 1900 hasta una fecha reciente. Este archivo fue depurado rigurosamente siguiendo las fases detalladas en la metodología.
\noindent
En la metodología, la Fase 2 se realiza utilizando un código desarrollado en Python para hacer el filtrado del archivo de sismos generales y para obtener los archivos individuales de cada región en estudio.
\noindent
El análisis de los archivos individuales de sismos y todos los cálculos relacionados al análisis estadístico se realizaron en R (versión 4.4.2) y RStudio (2024.09.1 Build 394), que proporcionaron herramientas específicas para los análisis frecuentes de datos.
\section{Resultados de estadísticos descriptivos}
\noindent
Para esta parte se ejecutaron los cálculos descritos en la Fase 4 de la metodología en RStudio, obteniéndose la Tabla \ref{tab:estadisticos_consolidados} con los resultados consolidados del análisis estadístico por regiones.
\noindent
\footnotesize
\centering
\begin{tabularx}{15.9cm}{|>{\centering\arraybackslash}p{2cm}|*{9}{>{\centering\arraybackslash}X|}} >{\centering\arraybackslash}p{1cm}|}
\caption{Estadísticos descriptivos consolidados.} \label{tab:estadisticos_consolidados}
\hline \rowcolor{gray!60}
\textbf{Región} & \textbf{Min} & \textbf{Max} .... [TRUNCATED] ...